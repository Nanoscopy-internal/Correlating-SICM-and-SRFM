
\subsection{Super-Resolved Fluorescence Microscopy}
\label{sec:srfm}
In order to understand how super-resolved fluorescence microscopy (SRFM)
circumvents the diffraction limit, it is helpful to recapitulate the operating
principle of fluorescence microscopy (FM). Fluorescence is the process of the
absorption of a photon by a molecule, the fluorophore, followed by the
spontaneous emission of a second photon. The transitions between the energetic
states of the fluorophore are depicted in
Fig.~\ref{fig:jablonski-fluorescence}a. Due to the non-radiative loss of
energy during the involved vibrational decays (black arrows in
Fig.~\ref{fig:jablonski-fluorescence}a), the energy $h\nu_\text{em}$ of the
emitted photon is smaller than the energy of the absorbed photon
$h\nu_\text{ex}$ (here, $h$ denotes the Planck constant). Since the frequency $\nu$
of a photon is given by $c\lambda^{-1}$ (with $c$: speed of light, $\lambda$:
wavelength of the photon), the wavelength $\lambda_\text{em}$ of the emitted
photon is larger than the wavelength $\lambda_\text{ex}$ of the absorbed
photon.      

\begin{figure}
%  \sidecaption
  \centering
  \includegraphics{gr/intro/Jablonski-fluorescence}
  
  \caption{%
    \textbf{Principle of fluorescence microscopy.}
    \textbf{a:}
      Jablonski diagram of fluorescence. A photon of energy
      $h\nu_\text{ex}$ gets absorbed and excites the molecule from its
      electronic ground state $S_0$ to a vibrational excited state of the
      electronic state $S_1$ (green arrow). After radiation-less relaxation
      (black arrow) into the vibrational ground state of $S_1$, the molecule
      returns to any vibrational state of $S_0$ (red arrow) by emitting a photon
      of energy $h\nu_\text{em}$, after which it relaxes to the lowest
      vibrational state of $S_0$.
    \textbf{b:}
      Typical absorbance (green) and
      emission (red) spectra of a fluorescent dye molecule. Green and red
      rectangles indicate the wavelength range used for excitation (exc) and
      detection.
  }
  \label{fig:jablonski-fluorescence}
\end{figure}

Typical absorbance/emission spectra of fluorophores used as dyes in FM are
shown in Fig.~\ref{fig:jablonski-fluorescence}b. In FM, a single wavelength or
a wavelength range (green rectangle Fig.~\ref{fig:jablonski-fluorescence}b) in
suitable to excite the fluorophore used is chosen, by using a suitable laser
or LED or by selecting the wavelength range via a band-pass filter from a
white light source and send through the objective lens of the microscope to
the sample. In most FMs (the so called epifluorescence microscopes), the
emission is collected via the same objective lens. Since the emission is
red-shifted with respect to the excitation wavelength, the emitted photons
(wavelength range depicted by red rectangle in
Fig.~\ref{fig:jablonski-fluorescence}b) can be separated from the ones used to
excite the fluorophores by a dichroic mirror.

\begin{figure}
  \includegraphics{gr/intro/FM-setups}
  \caption{\textbf{Beam paths and major components of fluorescent
      microscopes.} \textbf{a:} Excitation (green) and emission (red) beam
    paths of a wide-field microscope. \textbf{b:} Excitation (green) and
    emission beam paths (red) of a confocal microscope.  \textbf{c:} Emission
    beam path (filters omitted for clarity) of a confocal microscope showing
    how out-of-focus emission (dashed and dotted lines) is suppressed by the
    confocal pinhole.}
  \label{fig:FM-setups}
\end{figure}

The two most common implementations of FM used in the life sciences are the
wide-field and the confocal microscope. Both use a dichroic mirror to separate
excitation and emission, but their implementations differ in the way how the
sample is excited and how the emission is recorded. The beam paths and the
major components of a wide-field and a confocal FM are depicted in
Fig.~\ref{fig:FM-setups}. In a wide-field FM (Fig.~\ref{fig:FM-setups}a), the
entire sample is excited at 
the same time, and all fluorophores in the sample emit at the same time
(yellow circles in the right inset in Fig.~\ref{fig:FM-setups}a). The emission
of the fluorophores (solid red shape in Fig.~\ref{fig:FM-setups}a, only five
beam paths are shown for clarity) is collected by the objective lens, passes
the dichroic mirror and is further cleared by a band-pass filter. The emission
then is focused by the tube lens onto the camera, which records an image of
the emission of all fluorophores at the same time.

Each fluorophore in the sample generates a diffraction limited spot on the
camera, and spots which are not separated in space by at least the diffraction
cannot be distinguished as single spots. Thus, the wavelength which determines
the resolution in a wide-field microscope is the wavelength $\lamba_\text{em}$
of the emission.

The major disadvantage of a wide-field microscope is that the sample is not
only illuminated evenly in the focal plane of the objective lens, but also
fluorophores outside the focal plane in $z$-direction are excited and their
emission is collected (not shown for clarity). Since they are not located in
the focal plane of the objective lens, their emission does not result in a
sharp spot on the camera, but instead appears as a blurred background.    

This led to the invention of the confocal microscope
. The beam path of a confocal microscope is
shown in Fig.~\ref{fig:FM-setups}b. Here, the excitation light (green solid
shape) is focused into the sample by the objective lens and only the
fluorophores within the diffraction limited spot are excited. Thus, the
wavelength which determines the resolution of a confocal microscope is the
excitation wavelength $\lambda_\text{ex}$.  


% Ernst Abbe found in 1873 \cite{Abbe1873} that the resolution $d$ of
% conventional light microscopes is limited by diffaction to 
% \begin{equation}
%   d_\text{min} = \frac{\lambda}{2\mathrm{NA}} = \frac\lambda{2n\sin\alpha}\text.
%   \label{eq:abbe}
% \end{equation}
%  One consequence of the diffraction limit is
% that a beam of light cannot be focused to a spot smaller than $d_\text{min}$.

% The place in the micropscope where the diffraction limited spot limits the
% resolution of a microscopic recording depends on the implementation of the
% microscope.  

% \begin{figure}
% %  \includegraphics{gr/introduction/}
% \end{figure}

\subsection{General approach}

\subsubsection{Different approaches}
While the majority of SRFM techniques follows the approach outlined above,
other techniques exist that use different approaches.  


\subsection{Camera-based techniques}

\subsection{Scanning techniques}
%%% Local Variables:
%%% mode: latex
%%% TeX-master: "../manuscript"
%%% End:
