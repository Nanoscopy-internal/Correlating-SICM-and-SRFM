
\subsection{Fluorescence Microscopy}
\label{sec:fm}
In order to understand how super-resolved fluorescence microscopy (SRFM)
circumvents the diffraction limit, it is helpful to recapitulate the operating
principle of fluorescence microscopy (FM). Fluorescence is the process of the
absorption of a photon by a molecule, the fluorophore, followed by the
spontaneous emission of a second photon. The transitions between the energetic
states of the fluorophore are depicted in
Fig.~\ref{fig:jablonski-fluorescence}a. Due to the non-radiative loss of
energy during the involved vibrational decays (black arrows in
Fig.~\ref{fig:jablonski-fluorescence}a), the energy $h\nu_\text{em}$ of the
emitted photon is smaller than the energy of the absorbed photon
$h\nu_\text{ex}$ (here, $h$ denotes the Planck constant). Since the frequency $\nu$
of a photon is given by $c\lambda^{-1}$ (with $c$: speed of light, $\lambda$:
wavelength of the photon), the wavelength $\lambda_\text{em}$ of the emitted
photon is larger than the wavelength $\lambda_\text{ex}$ of the absorbed
photon.      

\begin{figure}
%  \sidecaption
  \centering
  \includegraphics{gr/intro/Jablonski-fluorescence}
  
  \caption{%
    \textbf{Principle of fluorescence microscopy.}  \textbf{a:} Jablonski
    diagram of fluorescence. A photon of energy $h\nu_\text{ex}$ gets absorbed
    and excites the molecule from its electronic ground state $S_0$ to a
    vibrational excited state of the electronic state $S_1$ (green
    arrow). After radiation-less relaxation (black arrow) into the vibrational
    ground state of $S_1$, the molecule returns to one of the vibrational
    states of $S_0$ (red arrow) by emitting a photon of energy
    $h\nu_\text{em}$, after which it relaxes to the lowest vibrational state
    of $S_0$.  \textbf{b:} Typical absorbance (green) and emission (red)
    spectra of a fluorescent dye molecule. Green and red rectangles indicate
    the wavelength range used for excitation (exc) and detection in FM.  }
  \label{fig:jablonski-fluorescence}
\end{figure}

Typical absorbance/emission spectra of fluorophores used as dyes in FM are
shown in Fig.~\ref{fig:jablonski-fluorescence}b. In FM, a single wavelength or
a wavelength range (green rectangle in Fig.~\ref{fig:jablonski-fluorescence}b) 
appropriate to excite the fluorophore used is chosen -- by using a suitable laser % I changed one "suitable" to "appropiate"
or LED or by selecting the wavelength range via a band-pass filter from a
white light source -- and send through the objective lens of the microscope to
the sample. In most FMs (the so called epifluorescence microscopes), the
emission is collected via the same objective lens. Since the emission is
red-shifted with respect to the excitation wavelength, the emitted photons
(wavelength range depicted by red rectangle in
Fig.~\ref{fig:jablonski-fluorescence}b) can be separated from the ones used to
excite the fluorophores by a dichroic mirror.

\begin{figure}
  \includegraphics{gr/intro/FM-setups}
  \caption{\textbf{Beam paths and major components of fluorescent
      microscopes.} \textbf{a:} Excitation (green) and emission (red) beam
    paths of a wide-field microscope. \textbf{b:} Excitation (green) and
    emission beam paths (red) of a confocal microscope.  \textbf{c:} Emission
    beam path (filters omitted for clarity) of a confocal microscope showing
    how out-of-focus emission (dashed and dotted lines) is suppressed by the
    confocal pinhole.}
  \label{fig:FM-setups}
\end{figure}

The two most common implementations of FM used in the life sciences are the
wide-field and the confocal microscope. Both use a dichroic mirror to separate
excitation and emission, but their implementations differ in the way how the
sample is excited and how the emission is recorded. The beam paths and the
major components of a wide-field and a confocal FM are depicted in
Fig.~\ref{fig:FM-setups}. In a wide-field FM (Fig.~\ref{fig:FM-setups}a), the
entire sample is illuminated at the same time, and all fluorophores in the sample
emit at the same time (yellow circles in the right inset in
Fig.~\ref{fig:FM-setups}a). The emission of the fluorophores (solid red shape
in Fig.~\ref{fig:FM-setups}a, only five beam paths are shown for clarity) is
collected by the objective lens, passes the dichroic mirror and is further
cleared by a band-pass filter. The emission then is focused by the tube lens
onto the camera, which records an image of the emission of all fluorophores at
the same time.

Each fluorophore in the sample generates a diffraction limited spot on the
camera, and spots which are not separated in space by at least the diffraction limit
cannot be distinguished as single spots. Thus, the wavelength which determines
the resolution in a wide-field microscope is the wavelength $\lambda_\text{em}$
of the emission.

The major disadvantage of a wide-field microscope is that the sample is not
only illuminated in the focal plane of the objective lens, but also
fluorophores above and below the focal plane are excited and their
emission is collected (not shown for clarity). Since they are not located in
the focal plane of the objective lens, their emission does not result in a
sharp spot on the camera, but instead appears as a blurred background.    

This led to the invention of the confocal microscope
\cite{Heimstaedt1911,Minsky1988}. The beam path of a confocal microscope is
shown in Fig.~\ref{fig:FM-setups}b. Here, the excitation light (green solid
shape) is focused onto the sample by the objective lens and only the
fluorophores within the resulting diffraction limited spot are excited. Thus, the
wavelength which determines the resolution of a confocal microscope is the
excitation wavelength $\lambda_\text{ex}$. In contrast to the wide-field
microscope, the emission of the fluorophores is focused onto a pinhole which
is located in front of a single detector (left inset in
Fig.~\ref{fig:FM-setups}b). Since this setup does not allow imaging the entire
sample at once, the sample has to be scanned with respect to the excitation
light beam and the image is collected pixel by
pixel. Figure~\ref{fig:FM-setups}c illustrates the major effect of the pinhole:
Emission from out-of-focus planes (dotted and dashed lines) is blocked by the
pinhole since it is located in the same focal plane as the sample (thus the
name \emph{confocal}). 

For a potential combination with SICM, the most important aspect of the two
techniques is the way the fluorescence is recorded. Wide-field FM uses a 
camera to record the entire sample at once, whereas confocal FM is a scanning
technique which records the single pixels subsequently, either by moving the
sample through the beam or vice versa.

\subsection{Circumventing the diffraction limit}
\label{sec:circumvent-diffraction-limit}
The general approach to circumvent the diffraction limit is to reduce the
fluorophores within a diffraction limited spot which contribute to the
recorded signal, that is, to turn the fluorophores into an ``off''-state. Only
the fluorophores which contribute to the signal remain in the ``on''-state.
However, a few more approaches exist, for example 4$\pi$-microscopy
\cite{Hell1994} or structured illumination microscopy (SIM)
\cite{Guerra1995}. The first uses two objectives on opposite sides of the
sample, which renders it impossible to be combined with SICM since the space
required for the scanning pipette is already occupied by the second
objective. In contrast, SIM is implemented on a wide-field microscope and
could thus be combined with SICM. Therefore, combining SIM and SICM is similar
to combining SICM with other wide-field based techniques (see
section~\ref{sec:smlm}). Since the gain in resolution is low compared with the
approaches explained in sections~\ref{sec:smlm}
and~\ref{sec:deterministic-approaches}, we refrain from explaining SIM in
detail here.


\subsubsection{Stochastic approaches}
\label{sec:smlm}
The stochastic approaches to SRFM use dye molecules which can be switched
between an inactivated (``off'') and an activated (``on'') state by light of a
specific wavelength. If only a sparse subset of molecules is activated, such
that, on average, only one dye molecule per diffraction limited spot emits,
the position of the emitting molecule can be localised with much better
precision than the diffraction limit. If this is repeated until all
fluorophores have emitted, plotting the positions instead of the emission of
all fluorophores yields an image with resolution beyond the diffraction
limit (localisation microscopy). The first techniques using this principle
were photo-activated localisation microscopy (PALM) \cite{Betzig2006} (using
the Eos fluorescent protein \cite{Wiedenmann2004} as marker), fluorescence
photo-activation localisation microscopy (FPALM) \cite{Hess2006} (using
photo-activatable GFP as marker), and stochastical
optical reconstruction microscopy (STORM) \cite{Rust2006} (using Cy5 and Cy3
dyes).

\begin{figure}
  \centering
  \includegraphics{gr/intro/localisation-mic}
  \caption{%
    \textbf{Principle of PALM.}
    \textbf{a:} Initially, all fluorophores are in the inactivated state (grey
    dots).
    \textbf{b:} A small subset of the fluorophores is activated (cyan
    background, white dots) and subsequently excited (green background, yellow
    dots) until the fluorophores bleach (black dots). The emission of the
    fluorophores is recorded. This series is repeated until all fluorophores
    have emitted.
    \textbf{c:} The positions of the fluorophores in the recordings are
    determined and plotted as a Gaussian distribution with a standard
    deviation equal to the uncertainty of the localisation and the single
    images are combined to a single image with improved resolution.
  }
  \label{fig:lm}
\end{figure}

As an example, the principle of PALM is illustrated in Fig.~\ref{fig:lm}. Initially,
all fluorophores within the sample are inactivated (``off''-state, grey dots
in Fig.~\ref{fig:lm}a). Next, light suitable to switch the fluorophores (cyan
background, Fig.~\ref{fig:lm}b) from the inactivated to the activated (``on'')
state (white dots) is applied. The intensity of the activation light is
selected such that only a small subset of the inactivated molecules is
activated, at maximum one molecule per area of the diffraction limited
spot. The activated molecules are excited (green background, yellow dots) and
their emission is recorded. The excitation light is applied as long as all
excited molecules have bleached (black dots). The series of activation,
excitation and read-out is repeated until all fluorophores have been excited
and bleached, which results in a set of images containing diffraction limited
emission spots of all fluorophores in the sample.

In a post-processing step (Fig.~\ref{fig:lm}c), the positions of the
fluorophores are determined by fitting (an approximation of) the instrument's
point spread function to the data. The positions are plotted as Gaussian
distributions with a standard deviation equal to the uncertainty of the
determination of the position by the fit, finally yielding an image with an
improved resolution (Fig.~\ref{fig:lm}c, right).

The resolution provided by this approach scales with the number of photons $N$
that are detected and theoretically is given by $\sigma_\text{PSF}/\sqrt{N}$
(with $\sigma_\text{PSF}$ denoting the standard deviation of the point spread
function of the microscope, which is in the range of 100\,nm). However, in
practice the resolution is in the range of 20\,nm \cite{Betzig2006,Rust2006}
since the theoretical resolution cannot be reached due to background, noise
and other factors. A more detailed description of the resolution enhancement
by localisation is provided by \cite{Mortensen2010}. 

This brief explanation is far from being exhaustive and does not consider the
differences in the implementations of the various techniques. An in-depth
overview is provided by several reviews of the techniques
\cite{Patterson2010,Sengupta2014,Liu2015,Sauer2017}, suitable dyes
\cite{Li2018a} as well as by detailed imaging protocols
\cite{Gould2009,Schermelleh2010,Linde2011}. However, one aspect of the
implementation which is important for the potential combination with SICM is
the illumination scheme used by these methods. Since wide-field illumination
is required to record the emission of the activated fluorophores, out-of-focus
emission occurs, which increases the background in the recordings and affects
the resolution. To avoid this, a total internal reflection (TIRF) illumination
scheme \cite{AMBROSE1956,Axelrod1981} is used, which limits the excitation to
the layer close to the interface of the microscopy slide and the sample.   


\subsubsection{Deterministic approaches}
\label{sec:deterministic-approaches}
In contrast to the stochastic approaches, which switch the fluorophores
randomly in space, the deterministic approaches use a single-spot illumination
and switch the fluorophores in the periphery of the excitation spot. The
underlying principle is called reversible saturable optical fluorescence
transitions (RESOLFT). Methods that use this principle are stimulated emission
depletion (STED) microscopy \cite{hell+wichmann,Klar2000}, ground state
depletion (GSD) microscopy \cite{Hell1995,Bretschneider2007} and RESOLFT with
photo-switchable proteins \cite{Hofmann2005}. The latter is often referred to
as RESOLFT microscopy, although STED and GSD use the RESOLFT principle, too. 

\begin{figure}
  \centering
  \includegraphics{gr/intro/direct-SR-mic}
  \caption{\textbf{Principle of STED microscopy.} \textbf{a:} Shape of
    excitation and ring-shaped depletion beam. \textbf{b:} The two beams are
    superimposed and the sample is scanned. Fluorophores in the center are
    excited and emit spontaneously (yellow dot), while the depletion beam
    induces stimulated emission of the fluorophores in the periphery (black
    dots). \textbf{c:} (left) Jablonski diagram of stimulated emission (SE);
    (center) Line profile of the intensity of the depletion beam (red) and
    probability (prob) to induce stimulated emission (black); (right)
    wavelengths of excitation (ex), detection (det) and stimulated emission
    (SE) superimposed on the absorbance/emission spectrum of a fluorescent
    dye.}
  \label{fig:STED-principle}
\end{figure}

As an example for RESOLFT, the principle of STED microscopy is sketched in
Fig.~\ref{fig:STED-principle}. The excitation spot
(Fig.~\ref{fig:STED-principle}a, top) is superimposed with a second, ring- or
doughnut-shaped spot (Fig.~\ref{fig:STED-principle}a, bottom). The
fluorophores are excited by the excitation light, but only those located in
the center of the minimum of the depletion beam are allowed to emit
spontaneously (yellow dot). The fluorophores at the periphery are de-excited
by the depletion beam and undergo stimulated emission (black dots). The
Jablonski diagram of stimulated emission is shown in
Fig.~\ref{fig:STED-principle}c (left). An incoming photon of the depletion
beam interacts with the excited fluorophore and causes it to de-excite to the
ground state and to emit a second photon of the same energy as the incoming
photon. Due to the non-linear relation between intensity and the probability
to induce stimulated emission and the shape of the depletion beam
(Fig.~\ref{fig:STED-principle}c, center), the size of the origin of
spontaneous emission in the center of the depletion beam is decreased. The
wavelength of the depletion beam and thus the wavelength of stimulated
emission is selected such that it is outside the detection window of the
microscope (Fig.~\ref{fig:STED-principle}c, right), which de-facto switches
fluorophores that undergo stimulated emission to an invisible ``off''-state.

In summary, this directly yields a recording of the fluorophores with a
resolution beyond the diffraction limit (Fig.~\ref{fig:STED-principle}d). The
resolution provided by STED microscopy depends on the intensity $I$ of the
depletion beam and is given by
\begin{equation}
  d_\text{min, STED} = \frac\lambda{2\mathrm{NA}\sqrt{I/I_\text{sat}}}\text,
  \label{eq:resolution-STED}
\end{equation}
which is a modification of Abbe's original equation
(Eq.~\ref{eq:diffraction-limit}). Here, $I_\text{sat}$ is the intensity
required to induce stimulated emission in half of the fluorophores. The best resolution (approximately 2\,nm) obtained by STED microscopy has
been recorded on samples containing fluorescent color centers in diamond
crystal \cite{Wildanger2012}.

Note that, by applying a depletion beam of a
different shape, STED microscopy allows to improve the resolution along the
optical axis \cite{Harke2008, Wildanger2009}. A large number of reviews articles exist
which summarize the use of STED (or in general, RESOLFT) microscopy in different
fields in the life sciences, much too much to be listed here. We would like to
refer the reader to exhaustive reviews, which, besides
reviewing various applications, detail technical aspects of STED microscopy
\cite{Turkowyd2016,Blom2017} as well as to a review which contains some
anecdotal sections on how STED was developed \cite{Sahl2019}.


% The first concept to circumvent the diffraction limit was published in 1994
% and realised in 2000 \cite{hell+wichmann,Klar2000}


% Ernst Abbe found in 1873 \cite{Abbe1873} that the resolution $d$ of
% conventional light microscopes is limited by diffaction to 
% \begin{equation}
%   d_\text{min} = \frac{\lambda}{2\mathrm{NA}} = \frac\lambda{2n\sin\alpha}\text.
%   \label{eq:abbe}
% \end{equation}
%  One consequence of the diffraction limit is
% that a beam of light cannot be focused to a spot smaller than $d_\text{min}$.

% The place in the micropscope where the diffraction limited spot limits the
% resolution of a microscopic recording depends on the implementation of the
% microscope.  

% \begin{figure}
% %  \includegraphics{gr/introduction/}
% \end{figure}

% \subsection{General approach}

% \subsubsection{Different approaches}
% While the majority of SRFM techniques follows the approach outlined above,
% other techniques exist that use different approaches.  




% \subsection{Scanning techniques}

%%% Local Variables:
%%% mode: latex
%%% TeX-master: "../manuscript"
%%% End:
