\section{Correlating SICM and Fluorescence Microscopy}
\label{sec:correlating-sicm-and-fm}
%%% Local Variables:
%%% mode: latex
%%% TeX-master: "../manuscript"
%%% End:
Correlative SICM and fluorescence microscopy opens up the opportunity to receive information, 
which are hidden and impossible to obtain by using one of this microscopy techniques alone. 
This is especially helpful for applications, where the reaction is not limited to one Untersuchungsobjekt, %german
for example when investigating the uptake of nanoparticles into cells. There it would be interesting to
untersuchen, sowohl %german
the cell membrane with a technique, which is able to do so $–$ AFM \cite{}, EM \cite{}, SICM \cite{Hansma1989} $-$, 
als auch %german
to follow the nanoparticle into the cell or to investigate simultaneously how the cytoskeleton 
react on the Zugabe %german
of the nanoparticle by using fluorescence microscopy methods.

% Den Teil brauchen wir ja vielleicht nicht. 
AFM, a method belonging to the scanning probe microscopy techniques, was successfully
combined to fluorescence microscopy techniques e.g. to investigate the topography of a 
cell or a part of a cell with the cytoskeleton \cite{Laishram2009,Liu2020}. But a big problem 
with this method is, that it is not the most favorable for investigating living cells in their
physiological surrounding. Certain destructive effects on the cell were found by using this 
technique \cite{}. A better fitting method for investigating living cells would be scanning 
ion conductance microscopy (SICM) \cite{Hansma1989}. 

% Hier jetzt der surface confocal scanning part, NP investigation part usw. (damit warte ich aber, bis ich weiß, wie das
% richtig eingefügt wird.

Nevertheless SICM does not necessary being used for investigating the topography of a cell, 
the nanopipette of a SICMic could be also used for the gezielte %german word! 
deposition of molecules, like fluorescently-labeled single stranded-DNA molecules 
\cite{Ying2002,Hennig2015}. The nanopipette could be also used for nanoinjection 
of fluorescently labeled molecules into living cells \cite{Hennig2015_nanoinj}. 
The SICM was therefore combined with wide-field and dSTORM microscopy, and mitochondria, 
actin and DNA could be successfully stained. One problem with the nanoinjection was that 
artifacts, due to the holes that were introduced into the membranes, could occur \cite{Hennig2015_nanoinj}.

SICM in combination with F\"orster resonance energy transfer (FRET) sensors could be used for
the localisation of {$\beta_{1}$}-adrenoceptors \cite{Wright2018}, {$\beta_{2}$}-adrenoceptors 
\cite{Nikolaev2010,Lyon2012,Wright2014,Wright2018} and {$\beta_{3}$-adrenoceptors \cite{Schobesberger2020} 
in healthy and impaired cardiomyocytes \cite{Wright2015,Berisha2017}. After measuring the 
topography (SICM) the nanopipette could be moved to places $-$ e.g. the z-grooves of the cell 
\cite{Gorelik2006,Lyon2009,Miragoli2011,Lyon2012,Rivaud2017} $-$ where the adrenoceptor distribution 
should be investigated. Then the pipette could be used for stimulating the receptor to procuce 
cAMP oder cGMP \cite{Xiang2003}, this could bind to the FRET sensor (in the cell), which changes 
its conformation and a FRET signal could be detected \cite{Nikolaev2010}.
