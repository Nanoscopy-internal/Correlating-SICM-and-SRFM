\section{Correlating SICM and Fluorescence Microscopy}
\label{sec:correlating-sicm-and-fm}
%%% Local Variables:
%%% mode: latex
%%% TeX-master: "../manuscript"
%%% End:
Correlative SICM and FM, despite the mismatch in resolution, allows to gather
information which cannot be obtained by using a single technique
alone. However, the degree of correlation differs, depending on the FM
technique combined with SICM. Combining SICM and wide-field FM allows to image
the same region of the sample, which, for example, can be used to find a
region of interest such as an area containg synapses in a culture of a
neuronal network \cite{Novak2013,Scheenen2015}. 

% Den folgenden Teil verstehe ich nicht wirklich, SICM kann ja auch 3D
% (allerdings natürlich nicht innerhalb von irgendetwas). Und irgendwie ist
% das ja auch nicht der "Haupt"-Vorteil, sondern, dass man überhaupt
% Fluoreszenz sehen kann.

% This is especially helpful for applications, where the reaction is not limited to \emph{one 
% dimension}, for example when investigating the uptake of nanoparticles into cells. There it would 
% be interesting to investigate, sowohl %german
% the cell membrane with a technique, which is able to do so -- AFM \cite{}, EM \cite{}, SICM 
% \cite{Hansma1989} --, als auch %german
% to follow the nanoparticle into the cell or to investigate simultaneously how the cytoskeleton 
% react on the addition (?) of the nanoparticle by using fluorescence microscopy methods.

% % Den Teil brauchen wir ja vielleicht nicht. 
% AFM, a method belonging to the scanning probe microscopy techniques, was successfully
% combined to fluorescence microscopy techniques e.g. to investigate the topography of a 
% cell or a part of a cell with the cytoskeleton \cite{Laishram2009,Liu2020}. But a big problem 
% with this method is, that it is not the most favorable for investigating living cells in their
% physiological surrounding. Certain destructive effects on the cell were found by using this 
% technique \cite{}. A better fitting method for investigating living cells would be scanning 
% ion conductance microscopy (SICM) \cite{Hansma1989}. 

% Hier jetzt der surface confocal scanning part, NP investigation part usw. (damit warte ich aber,
% bis ich weiß, wie das richtig eingefügt wird.

Topographical information obtained by SICM were successfully correlated with fluorescence signals
of stained cytoskeleton molecules obtained by laser scanning confocal microscopy in platelets 
\cite{Seifert2017}. 

In addition, hippocampal neurons, which were very delicate samples with many little protrusions, 
could be investigated with correlated SICM and confocal microscopy. With such a combination active
synaptic boutons could be identified \cite{Novak2013} or the topography could be correlated with 
the distribution (nicht das richtige Wort!?) of the mitochondria \cite{Takahashi2019}. 

Another very delicate sample, because of their three-dimensional, upright (?) structure are cilia.
Here SICM was used for investigating the topography of the cilia and the fluorescence was used to
distinguish between surface and sub-surface cilia, which could not be detected with SICM 
\cite{Zhou2018}.

Scanning surface confocal microscopy (SSCM) was a combination of scanning ion conductance and a 
scanning confocal microscopy \cite{Gorelik2002}. The topography and fluorescence were measured
simultaneously and -- because fluorescence signals should be investigated, which were originated
from particles at or near to the surface of the cell -- it was important to keep the distance 
between the sample and the pipette constant. In that study the SSCM was used for investigating the 
dynamics of the interaction of Cy3-labeled polyoma virus-like particles (VLPs) with Cos-7 cells. 
The SSCM could also be used for the investigation of the uptake of fluorescent nanoparticles 
(50~nm and 1~{\textmu}m) in immortalized human alveolar epithelial type 1 (AT1) cells and AT2 
cells \cite{Kemp2008} or for investigating the differentiation of stem cells \cite{Gorelik2008}.

In a study investigating the formation of human immunodeficiency virus (HIV) VLPs the correlated
SICM-fluorescence confocal microscopy (SICM-FCM) was used to investigate topography changes
together with die distribution of fluorescently labeled viral proteins like Gag or Viral Protein R 
(Vpr) \cite{Bednarska2020}. In contrast of investigating the virus formation with total internal 
reflection fluorescence (TIRF) microscopy \cite{Jouvenet2008} the upside (Oberseite?) of the cells 
were investigated. The release of VLPs from the upside can be a lot faster, than the release times 
that were known from TIRF measurements and difference between different cell types and labeled 
virus proteins were found \cite{Bednarska2020}. 

SSCM could be also used for investigating endocytotic pathways \cite{Shevchuk2008}. In a study 
investigating topography together with actin-binding protein-green fluorescent protein (Abp1-GFP)
and clathrin light-chain-enhanced green fluorescent protein (Clc-EGFP) an alternative closing 
mechanism for clathrin-coated pits (CCPs) \cite{Shevchuk2012}. In another study the uptake of
200~nm fluorescent, carboxylated latex particles into immortalized human alveolar epithelial type
1-like (AT1) cells was investigated. Additional to the fluorescence signal of the nanoparticle the 
fluorescence signal of actin-binding protein-green fluorescent protein (Abp1-GFP) and clathrin
light-chain-enhanced green fluorescent protein (Clc-EGFP) were detected. Both, actin, as part of
the cytoskeleton (Quelle), and clathrin, as one important actor of endocytosis (Quelle), give 
interesting Zusatzinformationen on how the nanoparticle is taken up into the cell 
\cite{Novak2014}. 

The study of Nashimoto was showing that for the detection of secretory proteins, like the von 
Willebrand factor, it is not always necessary to label the protein. Even with SICM alone the 
detection of such a protein was possible, which was verified by comparing fluorescence and SICM 
images \cite{Nashimoto2015}.

Nevertheless SICM does not necessary being used for investigating the topography of a cell, 
the nanopipette of a SICMic could be also used for the targeted deposition of molecules, like 
fluorescently-labeled single stranded-DNA molecules \cite{Ying2002,Hennig2015}. The nanopipette 
could be also used for nanoinjection of fluorescently labeled molecules into living cells 
\cite{Hennig2015a}. The SICM was therefore combined with wide-field and direct stochastic 
optical reconstruction microscopy (dSTORM), and mitochondria, actin and DNA could be successfully 
stained. One problem with the nanoinjection was that artifacts, due to the holes that were 
introduced into the membranes, could occur \cite{Hennig2015a}.

The delivery of fluorescently labeled molecules with the nanopipette could be also used for 
combining it with single-molecule diffusion investigations \cite{Bruckbauer2007, Bruckbauer2010}. 
In both studies the scanning ion conductance microscope was combined with TIRF investigations and 
the diffusion of single molecules over different regions at a spermatozoa were investigated. 

SICM in combination with Förster resonance energy transfer (FRET) sensors could be used for
the localisation of $\beta_{1}$-adrenoceptors \cite{Wright2018}, $\beta_{2}$-adrenoceptors 
\cite{Nikolaev2010,Lyon2012,Wright2014,Wright2018} and $\beta_{3}$-adrenoceptors 
\cite{Schobesberger2020} in healthy and impaired cardiomyocytes \cite{Wright2015,Berisha2017}. 
After measuring the topography (SICM) the nanopipette could be moved to places -- e.g. the 
z-grooves of the cell \cite{Gorelik2006,Lyon2009,Miragoli2011,Lyon2012,Rivaud2017} -- where the 
adrenoceptor distribution should be investigated. Then the pipette could be used for stimulating 
the receptor to produce cyclic adenosine monophosphate (cAMP) or cyclic guanosine monophosphate 
(cGMP) \cite{Xiang2003}, this could bind to the FRET sensor (in the cell), which changes its 
conformation and a FRET signal could be detected \cite{Nikolaev2010}.

The SICM-nanopipette could be also used for investigating membrane rafts (Quelle) by coordinated 
manipulation of focal adhesions. The pipette was functionalized with fibronectin, gangliosides
were fluorescently labeled, the pipette deformed the focal adhesion side and the behavior of the 
fluorescently labeled gangliosides were detected. This was seen as a hint, how membrane rafts 
reacted \cite{Fuentes2012}. 

Another example for a SICM-nanopipette based manipulation could be found in the study on the 
generation and distribution of plasma-derived extracellular vesicles (PEVs) on MDA-MB-231 and U87 
cells. Here the nanopipette were used for both, investigating the topography and for inducing the 
generation of the PEVs. Additionally the SICM was build up onto a confocal microscope, which could 
be used for the detection of Calcium \cite{Wang2020}.

% Dieser Abschnitt passt nicht so wirklich zu Fluoreszenz-Mikroskopie! Vielleicht noch so was 
% schreiben, wie „SICM kann nicht nur mit Fluoreszenz mikroskopischen Methoden kombiniert werden, 
% sondern kann auch mit … kombiniert werden!? Sonst kann der einfach auch weg. Mal gucken.

By using a special illumination system consisting of a light emitting diode (LED) light ring, SICM 
could also being combined with condenser-free phase contrast, darkfield and brightfield images and 
even Rheinberg illuminations were possible \cite{Webb2014}.

% Abschnitt weitere Korrelationen (indirekte)

Beside these more or less \emph{direct} combinations and correlations, which are displaying images 
obtained by SICM and fluorescence microscopy of the same cell/same region, also \emph{indirect} 
correlations exist. They are showing images under comparable conditions and extent the 
investigation to aspects, which could not be obtained by using one technique alone. Such 
\emph{indirect} correlations were used for investigating the uptake of nanotubes \cite{Lee2013}, 
nanoparticles \cite{Gesper2017} or for investigating healthy and failing myocytes \cite{Lyon2009}. 
Nichtsdestotrotz (german)
is it important to mention, that they could just giving hints of correlations, it 
is not directly comparable with investigations of both in the same cell. 

%%% Local Variables:
%%% mode: latex
%%% TeX-master: "../manuscript"
%%% End:
