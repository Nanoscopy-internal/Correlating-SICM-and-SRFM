\section{Correlating SICM and Fluorescence Microscopy}
\label{sec:correlating-sicm-and-fm}
%%% Local Variables:
%%% mode: latex
%%% TeX-master: "../manuscript"
%%% End:
Nevertheless SICM does not necessary being used for investigating the topography of a cell, 
the nanopipette of a SICMic could be also used for the gezielte %german word! 
deposition of molecules, like fluorescently-labeled single stranded-DNA molecules 
\cite{Ying2002,Hennig2015}. The nanopipette could be also used for nanoinjection 
of fluorescently labeled molecules into living cells \cite{Hennig2015_nanoinj}. 
The SICM was therefore combined with wide-field and dSTORM microscopy, and mitochondria, 
actin and DNA could be successfully stained. One problem with the nanoinjection was that 
artifacts, due to the holes that were introduced into the membranes, could occur \cite{Hennig2015_nanoinj}.

SICM in combination with F\"orster resonance energy transfer (FRET) sensors could be used for
the localisation of {$\beta_{1}$}-adrenoceptors \cite{Wright2018}, {$\beta_{2}$}-adrenoceptors 
\cite{Nikolaev2010,Lyon2012,Wright2014,Wright2018} and {$\beta_{3}$-adrenoceptors \cite{Schobesberger2020} 
in healthy and impaired cardiomyocytes \cite{Wright2015,Berisha2017}. After measuring the 
topography (SICM) the nanopipette could be moved to places - e.g. the z-grooves of the cell 
\cite{Gorelik2006,Lyon2009,Miragoli2011,Lyon2012,Rivaud2017} - where the adrenoceptor distribution 
should be investigated. Then the pipette could be used for stimulating the receptor to procuce 
cAMP oder cGMP \cite{Xiang2003}, this could bind to the FRET sensor, which changes 
its conformation and a FRET signal could be detected \cite{Nikolaev2010}.
