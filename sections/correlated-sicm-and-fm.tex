\section{Correlating SICM and Fluorescence Microscopy}
\label{sec:correlating-sicm-and-fm}
%%% Local Variables:
%%% mode: latex
%%% TeX-master: "../manuscript"
%%% End:
Nevertheless SICM does not necessary being used for investigating the topography of a cell, the nanopipette of a SICMic could be also used for the gezielte %german word! 
deposition of molecules, like fluorescently-labeled single stranded-DNA molecules \cite(Ying2002,Hennig2015). The nanopipette could be also used for nanoinjection of fluorescently labeled molecules into living cells \cite(Hennig2015_nanoinj). The SICM was therefore combined with wide-field and dSTORM microscopy, and mitochondria, actin and DNA could be successfully stained. One problem with the nanoinjection was that artifacts, due to the holes that were introduced into the membranes, could occur \cite(Hennig2015_nanoinj).
