\section{Correlating SICM and Fluorescence Microscopy}
\label{sec:correlating-sicm-and-fm}

% Assume a figure here which shows the different degrees of correlation
\begin{figure}
  \includegraphics[width=\textwidth]{gr/correlating/sicm+fm}
  \caption{%
    \textbf{Combining SICM and wide-field fluorescence microscopy.}
    \textbf{a:} Sketch of the data obtained by combining SICM and wide-field
    FM. 
    \textbf{b:} (left) FM image of a cilium in a RPE-1 cell, (right) SICM recording of
    the area marked by the dashed rectangle. The cilium is not visible since
    it is located benaeth the cell membrane \textbf{c:} (left) FM image of a
    cilium in a MDCK cell, (right) SICM recording of the
    region marked by the dashed rectangle. The cilium is clearly visible since
    it protrudes into the extracellular space.
    \textbf{b, c:} Reprinted with permission from Yuanshu Zhou, Masaki Saito,
    Takafumi Miyamoto, et al.: \emph{Nanoscale Imaging of Primary Cilia with
      Scanning Ion Conductance Microscopy.} Anal. Chem. 2018, 90, 4,
    2891–2895. Copyright 2018 American Chemical Society.
  }
  \label{fig:sicm+fm}
\end{figure}

Correlative SICM and FM, despite the mismatch in resolution, allows to gather
information which cannot be obtained by using a single technique
alone. However, the degree of correlation as well as the technical
implementation differs, depending on the FM technique combined with SICM, as
we will detail in the following sections.

Although recording SICM and FM data independently on different samples under
the same conditions can support conclusions that could not be drawn from data
from a single technique alone \cite{Gesper2017,Lee2013,Lyon2009}, we will
focus on applications were SICM and FM data has been recorded from the same
sample. Furthermore, we focus on far-field FM techniques and do not review the
combination of SICM with nearfield optical microscopy  
\cite{Korchev2000,Shevchuk2001,Rothery2003,Bruckbauer2002} or studies where
the SICM pipette is used for targeted deposition 
\cite{Ying2002,Bruckbauer2007,Bruckbauer2010,Hennig2015} or for injection 
\cite{Hennig2015a} of molecules, as a nano-sensor \cite{Piper2006} or for manipulating the sample
\cite{Fuentes2012,Wang2020}.

% Vielleicht können wir die letzten beiden Sätze (siehe Absatz darüber) noch einfügen, 
% dann würde das so aussehen, dass die Entscheidung, über diesen Bereich nicht zu berichten 
% gezielt war und nicht dass wir sie vergessen hätten!?

\subsection{Combining SICM and wide-field FM}
\label{sec:SICM+widefield}
Combining SICM and wide-field FM allows to image the same region of the sample
with both techniques (Fig.~\ref{fig:sicm+fm}a). Due to the different fields of
view, the different pixel sizes and the different times required to record one
image, the two data sets are not recorded synchronously and cannot be
correlated pixel-by-pixel. Technically, the combination is simple, since
the SICM only has to be mounted onto an inverted optical FM and both
instruments can be controlled independently by their own software.

So far, the combination of SICM and wide-field FM has been used to find a
region of interest such as an area containing active synapses in a culture of
a neuronal network \cite{Scheenen2015} or to
identify the neuron which had been investigated by SICM-guided patch-clamp of
synaptic boutons \cite{Novak2013}.

Furthermore, it has been used to first identify primary cilia in a cell
culture and subsequently use SICM to reveal whether the selected cilium was
extending into the extracellular space or was embedded in the cell
(Fig.~\ref{fig:sicm+fm}b, c), an
information which could not be retrieved by using FM alone \cite{Zhou2018}.

Cardio-myocytes exhibit a unique topography with transverse (T-)tubules and
crests. By first recording the topography
and subsequently moving the SICM pipette to either a T-tubule or a crest,
followed by the local application of agonists of either the $\beta_1$- or
$\beta_2$-adrenergic receptor (AR) and recording the intensity ratio of cyan
and yellow fluorescent protein, coupled to a Förster resonance energy transfer
(FRET) based reporter of cyclic AMP, it was
shown that the distribution of the $\beta_2$-AR was altered in failing
cardio-myocytes \cite{Nikolaev2010}. Furthermore, this approach was used to
investigate the effect of partial mechanical unloading, showing that it
reduces the cyclic AMP response of $\beta_2$-ARs at the T-tubules
\cite{Wright2018}. To investigate the distribution of the $\beta_3$-AR, the
approach was extended to use a FRET-based reporter of cyclic GMP, showing that
a redistribution of the $\beta_3$-AR occurs in failing cardio-myocytes, too
\cite{Schobesberger2020}.



\subsection{Combining SICM and confocal FM}

\begin{figure}
  \includegraphics[width=\textwidth]{gr/correlating/sicm+cfm2}
  \caption{%
    \textbf{Combining SICM and confocal fluorescence microscopy.}
    \textbf{a:} Schematic illustration of the data acquired by combining SICM and confocal FM.
    \textbf{b:} 
    	(left:) Topography of a neuronal network.
    	(center:) Fluorescence of the same neuronal network as depicted on the left.
    	(right:) Overlay of topography and fluorescence of two regions of interest (ROI 1, ROI 2) from
    	the left and middle image.
    Reprinted with permission from P. Novak, J. Gorelik, U. Vivekananda, et
    al.: \emph{Nanoscale-Targeted Patch-Clamp Recordings of Functional
      Presynaptic Ion Channels.} Neuron 2013, 79, 1067--1077.
  }
  \label{fig:sicm+cfm}
\end{figure}

Similar to a SICM and wide-field combination, a combination of SICM and
confocal FM enables imaging of the same region and sample with both techniques
(Fig.~\ref{fig:sicm+cfm}a). The technical implementation involves mounting a
SICM onto an inverted confocal FM. The software used to control the two
microscopes does not have to be the same as both microscopes record data
independently from one another. However, as the recordings are not
simultaneous, pixel-by-pixel correlation is hardly, if at all, possible.

SICM and confocal FM have been combined for the first time by Korchev et al.
\cite{Korchev2000} in 2000. The group used this combination to verify the
volume measurements obtained by SICM. Just a year later, further development
of this combination enabled simultaneous acquisition of Ca\textsuperscript{2+}
imaging and topography data \cite{Shevchuk2001}.

Seifert et al. \cite{Seifert2017} investigated the morphological activity of
platelets by time-lapse SICM and subsequently obtained confocal recordings of
cytoskeletal proteins of the investigated cells, allowing them to correlate
the activity of the platelets and the structure of the cytoskeleton.
In co-cultures of cardio-myocytes and myofibroblasts, the dynamics of
myocyte-myofibroblast connections was linked to the expression of connexin-43
\cite{Schultz2019}.


SICM has been shown to allow patch-clamp recordings from tiny neuronal
structures like synaptic boutons \cite{Novak2013}. The combination of SICM and
confocal FM allowed to first record an overview image and to identify active
synaptic boutons (Fig.~\ref{fig:sicm+cfm}b), which, in a second step, could be
patch-clamped.

A second study on neurons investigated, besides others, cargo transport in
neuronal dendrites \cite{Takahashi2019}. Here, the combination of
confocal FM and SICM provided a hint that the transport processes observed by
SICM were mediated by mitochondria.


In a study investigating the formation of human immunodeficiency virus-like
particles (VLPs) the correlated SICM-fluorescence confocal microscopy was used
to investigate topography changes together with the distribution of
fluorescently labeled viral proteins like Gag or Viral Protein R
\cite{Bednarska2020}. In contrast to investigating the virus formation with
TIRF microscopy \cite{Jouvenet2008}, the combination of SICM and confocal FM
allowed to investigate the apical side of the cells. It was found that the
release of VLPs is a lot faster than the release at the basal cell side as
known from TIRF investigations and furthermore, a potential impact on the
release time by the application of a fluorescent marker was found
\cite{Bednarska2020}.

The applications reviewed above used only a single confocal layer. Yet,
a combination of SICM and $z$-stacking confocal FM could be used to yield more
correlating data. After imaging the entire volume that is scanned by SICM, the
data could be aligned and non-correlating data could be separated from
correlating data. If stage scanning would be omitted, the two measurements
could be performed simultaneously, but not synchronously.  


\subsection{Surface scanning}

Scanning surface confocal microscopy (SSCM) \cite{Gorelik2002a} is a
combination of SICM and scanning confocal microscopy (SCM) which enables
simultaneous SICM and FM imaging of the surface of the same sample. In SSCM,
the fluorescence data for each pixel is recorded at the position determined by
SICM for that pixel. Up to now, two slightly different setups have been proposed
(Fig.~\ref{fig:sscm_setup}) \cite{Gorelik2002a,Shevchuk2013}.

\begin{figure}
  \centering
  \includegraphics{gr/correlating/sscm_setup}

  \caption{%
    \textbf{Two setups for a SSC microscope.}
    \textbf{a:} Diagram of the combined setup with an $x,y,z$ sample scanner.
    \textbf{b:} Diagram of the combined setup with a $x,y$ sample scanner and
    additional $z$-scanners for probe and objective. 
  }
  \label{fig:sscm_setup}
\end{figure}

In both setups, a SICM is mounted onto an inverted confocal microscope and
before simultaneous recording, the focus of the confocal FM objective is
aligned with the tip of the SICM probe via micrometer screws and the confocal
microscope's focusing mechanics. The
difference between the two setups is the way how the SICM pipette, the
confocal focus and the sample are positioned with respect to each other during
imaging. The
first setup (Fig.~\ref{fig:sscm_setup}a) keeps the 
SICM probe at a fixed position and scanning is performed by the sample
stage which positions the sample in all three dimensions.

In contrast, the second setup (Fig.~\ref{fig:sscm_setup}b) uses the sample
stage to position the sample only in $x$- and $y$-direction while objective
and probe are positioned by additional piezo actors in $z$-direction. Thus,
when the surface is detected by the SICM pipette, the position of the confocal
spot has to be adjusted by the objective positioning piezo actor subsequently.

Both setups do not allow to use the controlling software of the confocal
microscope since the fluorescence data has to be recorded after the SICM
pipette has detected the surface. Thus, the SICM controlling soft- and
hardware had to be extended to at least allow reading from the detectors of
the confocal microscope. 

Switching from three-axis stage scanning to $z$-movement of the pipette and
objective allowed for faster scan rates since the resonant frequency of the 3D
scanning stage limited the recording velocity. Additionally, usage of three
piezo-actors on the same platform led to cross-talk and consequentially
artefacts in the SICM data \cite{Shevchuk2013}.


Both setups keep the probe-objective distance constant and enable the
acquisition of a fluorescence image and the topography of the sample surface in
the same scan.

\begin{figure}
  \includegraphics[width=\textwidth]{gr/correlating/sicm+scm}
  \caption{%
    \textbf{Combining SICM and SCM.}
    \textbf{a:} Schematic illustration of the correlating data acquired by SSCM.
    \textbf{b:} SSCM recordings of fixed clathrin-GFP transfected Cos-7 cells. 
    	(left:) Cell surface topography. 
    	(center:) Fluorescence recording of clathrin in the cell shown left. 
    	(right:) 3D overlay of topographical and fluorescent recordings. 
    Reprinted from Shevchuk, A.I., Hobson, P., Lab, M.J., et al.: \emph{Endocytic
    pathways: combined scanning ion conductance and surface confocal
    microscopy study.} Pflugers Arch - Eur J Physiol 2008, 456, 227–235 with permission.
  }
  \label{fig:sicm+scm}
\end{figure}

As the SICM and SCM data sets are recorded simultaneously, all data points
inherently have a correlating data point in the other set, as illustrated in
Fig.~\ref{fig:sicm+scm}a. This allows for determining the distribution of
proteins in the proximity of the cell membrane (Fig.~\ref{fig:sicm+scm}b).



First, SSCM has been used for investigating the dynamics of the interaction of
Cy3-labeled polyoma virus-like particles (VLPs) with Cos-7 cells
\cite{Gorelik2002a}.

Furthermore, it has been utilized in studies investigating endocytotic
pathways \cite{Shevchuk2008a} (Fig.~\ref{fig:sicm+scm}b), besides others
revealing an alternative closing mechanism for clathrin-coated pits by
recording topography simultaneously with the fluorescence of tagged
actin-binding protein and the tagged clathrin light-chain
\cite{Shevchuk2012}. A further study on clathrin-mediated endocytosis revealed
the effect of mutations in the Dynamin~2 gene on the maturation and
internalisation of clathrin coated pits \cite{Ali2019}.

\begin{figure}
  \centering
  \includegraphics{gr/correlating/SSCM-np}
  \caption{%
    \textbf{Investigation on nanoparticle uptake by SSCM.}  \textbf{a:} Height
    profile of a section of a AT1-like cell treated with 200\,nm fluorescent
    nanoparticles.  \textbf{b:} Slope representation of the area shown in a.
    \textbf{c:} Surface fluorescence of the same area.  \textbf{d:} Overlay
    of surface fluorescence and slope representation. Yellow arrowheads mark
    nanoparticles that co-localise with membrane protrusions (white
    arrowheads), blue arrowheads mark nanoparticles being in the process of
    internalisation, green arrowheads mark fully internalised nanoparticles.
    Reprinted with permission from P. Novak, A Shevchuk, P. Ruenraroengsak, et
    al.: \emph{Imaging Single Nanoparticle Interactions with Human Lung Cells
      Using Fast Ion Conductance Microscopy.} Nano Lett. 2014, 14, 3,
    1202–1207. Copyright 2014 American Chemical Society.}
  \label{fig:sscm-np}
\end{figure}


SSCM was used to characterise the nanoparticle uptake by primary human alveolar
epithelial type 2 (AT2) and by immortalized AT2 cells. Only the latter showed
a strong tendency to internalise nanoparticles. Since, furthermore, the
immortalized cells exhibited several characteristics of alveolar
epithelial type 1 (AT1) cells, it was suggested that AT1 cells play a pivotal
role in the translocation of nanometer-sized particles in the lung
\cite{Kemp2008}.

Figure~\ref{fig:sscm-np} shows nanoparticles in the proximity of the surface of
AT1-like cells at three different stages of the internalisation: co-localised
with membrane protrusions (yellow and white arrowheads, respectively), in the
process of internalisation (blue arrowheads) and fully internalised (green
arrowheads). An improvement of the scanning velocity by the addition of a stiff,
short-range piezo actor onto the pipette positioning piezo actor allowed to
follow the internalisation process with a temporal resolution of 15 seconds in
detail \cite{Novak2014}.

Furthermore, SSCM was used for investigating the differentiation of stem cells
\cite{Gorelik2008} and the correlation of topography and T-tubule distribution
\cite{Lyon2012} or topography and mitochondria localisation and appearance
\cite{Miragoli2016} in healthy and heart-failure cardiomyocytes.

%%% Local Variables:
%%% mode: latex
%%% TeX-master: "../manuscript"
%%% End:
