\section{Correlating SICM and Super-Resolved Fluorescence Microscopy}
\label{sec:correlating-sicm-and-srfm}
%%% Local Variables:
%%% mode: latex
%%% TeX-master: "../manuscript"
%%% End:

For resolving tiny structures like single cytoskeleton fibers, nanoparticle, with sizes smaller
than the diffraction-limit, or protein complexes, it would be helpful to use microscopy
techniques, which can resolve structures beyond the diffraction limit.

% Hier füge ich jetzt einen Abschnitt zum diffraction limit ein. Ich habe gesehen das du (Patrick) 
% das in deiner SRFM-Einleitung nur im Kommentar eingefügt hattest.

The diffraction limit was defined by Ernst Abbe in 1873 \cite{Abbe1873} and could be described 
with the following equation.

\begin{equation}
d = \frac{\lambda}{2 n \sin{\alpha}}
\label{eq:Abbe_diff}
\end{equation}

In this equation $\lambda$ is the wavelength of light, $n$ the Brechungsindex (german) of the 
medium between objective and sample and $\alpha$ is half the opening Winkel (german) of the 
objective. Following this equation and under ideal conditions, the limit for structures, that 
could be resolved, would be halftimes the wavelength.

One of that techniques, that can resolve structures beyond the diffraction limit, is STED 
microscopy (see section~\ref{sec:deterministic-approaches}). In a proof-of-principle study the 
topography of a group of fixed HeLa cells together with the distribution of actin filaments was
investigated \cite{Hagemann2018}. Although both methods were able to investigate living cells, the 
cells were fixed in that study, because the investigation was done consecutively onto two 
microscopes. The topography was measured with SICM and the distribution of actin with STED 
microscopy (see figure~\ref{fig:SICM_STED1}).

% Hier soll jetzt Figure 4 unserer ACSNano-Publikation folgen. Patrick, du hast doch die 
% Ursprungs-/Original-PDF-Version davon, oder? Kannst du die dann bitte hier einfügen und 
% natürlich gucken, dass wir die Rechte dafür bekommen (das wolltest du ja machen, damit wir das 
% alles in einem Dokument haben).

\begin{figure}	
  \centering
    \includegraphics[clip,scale=1]{.PDF} % @Patrick hier bitte den Filename eintragen
      \caption{
      \textbf{Correlative SICM and STED recording of a group of HeLa cells.} 
      (\textbf{Aa}) SICM recording of the HeLa cells, (\textbf{Ab}) slope representation of the
      SICM recording. Confocal (\textbf{Ba}) and STED (\textbf{Bb}) recording of the region marked 
      with a black dotted line in \textbf{Aa} and \textbf{Ab}. (\textbf{C}) Relative position of 
      the SICM (yellow rectangle) and the confocal/STED (gray rectangle) images against each 
      other. Magnification of the STED (\textbf{Da}), confocal (\textbf{Db}), SICM (\textbf{Dd}) 
      and slope (\textbf{De}) images. The region, that is magnified, is marked with a white 
      arrowhead in \textbf{Aa, Ab, Ac and Ad}. (\textbf{Df} and (\textbf{Dg}) show an overlay of 
      the STED with the SICM (\textbf{Df}) or the slope representation (\textbf{Dg}). 
      (\textbf{Dc}) Height profile along the white dotted line in \textbf{Dd}. \newline Reprinted 
      with permission from \emph{Philipp Hagemann*, Astrid Gesper*, Patrick Happel}, Correlative 
      stimulated emission depletion and scanning ion conductance microscopy, \emph{ACS Nano} 2018, 
      12, 5807-5815, DOI: 10.1021/acsnano.8b01731. Copyright 2020 American Chemical Society}
  \label{fig:SICM_STED1}
\end{figure}
