\section{Correlating SICM and Super-Resolved Fluorescence Microscopy}
\label{sec:correlating-sicm-and-srfm}
%%% Local Variables:
%%% mode: latex
%%% TeX-master: "../manuscript"
%%% End:

For resolving tiny structures like single cytoskeleton fibers, nanoparticle, with sizes smaller
than the diffraction-limit, or protein complexes, it would be helpful to use microscopy
techniques, which can resolve structures beyond the diffraction limit of light (see 
equation~\ref{eq:diffraction-limit}; \cite{Abbe1873}).

% Hier vielleicht auch noch Fig. 3 aus der ACSNano-Publikation einfügen??? Würde vielleicht ganz
% gut passen!?

One of that techniques, that can resolve structures beyond the diffraction limit, is STED 
microscopy (see section~\ref{sec:deterministic-approaches}). In a proof-of-principle study the 
topography of a group of fixed HeLa cells together with the distribution of actin filaments was
investigated \cite{Hagemann2018}. Although both methods were able to investigate living cells, the 
cells were fixed in that study, because the investigation was done consecutively onto two 
microscopes. The topography was measured with SICM and the distribution of actin with STED 
microscopy (see figure~\ref{fig:SICM_STED1}).

% Hier soll jetzt Figure 4 unserer ACSNano-Publikation folgen. Patrick, du hast doch die 
% Ursprungs-/Original-PDF-Version davon, oder? Kannst du die dann bitte hier einfügen und 
% natürlich gucken, dass wir die Rechte dafür bekommen (das wolltest du ja machen, damit wir das 
% alles in einem Dokument haben).

\begin{figure}	
  \centering
%    \includegraphics[clip,scale=1]{.PDF} % @Patrick hier bitte den Filename eintragen
      \caption{
      \textbf{Correlative SICM and STED recording of a group of HeLa cells.} 
      (\textbf{Aa}) SICM recording of the HeLa cells, (\textbf{Ab}) slope representation of the
      SICM recording. Confocal (\textbf{Ba}) and STED (\textbf{Bb}) recording of the region marked 
      with a black dotted line in \textbf{Aa} and \textbf{Ab}. (\textbf{C}) Relative position of 
      the SICM (yellow rectangle) and the confocal/STED (gray rectangle) images against each 
      other. Magnification of the STED (\textbf{Da}), confocal (\textbf{Db}), SICM (\textbf{Dd}) 
      and slope (\textbf{De}) images. The region, that is magnified, is marked with a white 
      arrowhead in \textbf{Aa, Ab, Ac and Ad}. (\textbf{Df} and (\textbf{Dg}) show an overlay of 
      the STED with the SICM (\textbf{Df}) or the slope representation (\textbf{Dg}). 
      (\textbf{Dc}) Height profile along the white dotted line in \textbf{Dd}. \newline Reprinted 
      with permission from \emph{Philipp Hagemann*, Astrid Gesper*, Patrick Happel}, Correlative 
      stimulated emission depletion and scanning ion conductance microscopy, \emph{ACS Nano} 2018, 
      12, 5807-5815, DOI: 10.1021/acsnano.8b01731. Copyright 2020 American Chemical Society}
  \label{fig:SICM_STED1}
\end{figure}

The topography of a group of more than four fixed HeLa cells can be seen in 
figure~\ref{fig:SICM_STED1}Aa the corresponding slope representation in Ab, the confocal image in 
Ac and the STED image in Ad. The confocal and STED image represents the same region as marked with 
a black dotted line in Aa and Ab. C is showing the arrangement of the SICM image (yellow 
rectangle) against the confocal/STED image (gray rectangle). Many different cell topographies were
visible in Aa. 

A round cell with a maximum height of approximately 16.4~{\textmu}m height can be seen in the 
right upper region of the image (marked with a purple arrow in Aa/Ab/Ac/Ad). The cell next to this 
was flat and exceeding the scanned region (yellow arrow). A second flat but smaller cell was 
marked with a white arrow, which was ending with a white dot. Marked with an orange arrow (Aa, Ab) 
respectively with a cyan arrow (Ba, Bb) was a small, kind of roundish cells. This cell posesses 
a protrusion (white arrowhead), those magnification could be seen in Dd/Df respectively in De/Dg. 
The protrusion was approximately 4~{\textmu}m long and 5~{\textmu}m high (Dc, 
\cite{Hagemann2018}).

Beside the different topographies of these two cells, they also display different types of actin 
fibers. In the inside of the first mentioned round cell nearly no actin signal could be detected 
(some dot-like structures were marked with purple arrowheads), whereas at the borders of the cell 
some hair-like structures were visible (white arrow). The long, flat cell showed some long actin 
fibers (yellow arrowheads). The cell marked with the white arrow, which was ending with a white 
dot, could be just seen half in the confocal/STED images and dot-like actin structures 
predominate. The cell with the protrusion was showing fan-like actin structures (cyan arrow, which
was ending with a cyan dot and white arrow, which was ending with a white dot) where the cell was
connected to surrounding cells and actin agglomerations (cyan arrow). The agglomerations turned 
out to be \emph{rings} by watching the STED image (cyan circle). Such a structure could be also 
found in the protrusion (white arrowhead, Df and Dg). In the center of the \emph{actin ring} two 
dots could be identified (black arrowheads, Da).

Structures that would have been missed and/or could have been misinterpreted, if just one single
technique alone would have been used:

\noindent\textsf{
\begin{tabular}[c]{p{6cm}|p{3cm}|p{3.75cm}}
\label{tab:missing_structures}
What? & visible in fig.~\ref{fig:SICM_STED1} by SICM & visible in fig.~\ref{fig:SICM_STED1} by STED microscopy \\ \hline
The connections between the small round cell with the protrusion and the other cells & \textbf{PARTLY} & YES \\ \hline
The full extent of the protrusion & YES & \textbf{NO} \\ \hline
The connection between the roundish and the flat cell & \textbf{NO} & YES \\
\end{tabular}}

This points out again one of the biggest advantage of correlative microscopy: making things 
visible, that were invisible to one of the microscopy techniques alone. This could be due to 
fluorescence signals that were out of focus -- this could be the case by the actin signal that
suddenly ends in the protrusion -- (STED microscopy) or due to overhanging membranes -- this 
could be the case by the connections between the cells (especially between the highest and the
flat cell) -- (SICM) (Quelle).

As mentioned in section~\ref{sec:correlating-sicm-and-fm}, SICM was also successfully combined 
with dSTORM (see section~\ref{sec:smlm}) \cite{Hennig2015_nanoinj}. The difference to the 
SICM/STED microscopy correlation study was, that the SICM was in that study used for nanoinjecting 
fluorescently labeled molecules -- ATTO 655-phalloidin -- into living U2OS cells. 

% Hier jetzt noch was zu SICM/SNOM!

%%% Local Variables:
%%% mode: latex
%%% TeX-master: "../manuscript"
%%% End:
