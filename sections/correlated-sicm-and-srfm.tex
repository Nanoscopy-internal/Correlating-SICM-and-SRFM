\section{Correlating SICM and Super-Resolved Fluorescence Microscopy}
\label{sec:correlating-sicm-and-srfm}
%%% Local Variables:
%%% mode: latex
%%% TeX-master: "../manuscript"
%%% End:

In contrast to the combination of AFM and SRFM, the development of correlated
SICM and SRFM is still in its very early stages. The first combination of SRFM
and SICM has been the use of STORM to determine the number of fluorescent
molecules deposited via an SICM tip \cite{Hennig2015}. However, recently a
first proof-of-principle study combining STED and SICM has been published
\cite{Hagemann2018}.

In the latter, the topography of a group of fixed HeLa has been investigated by SICM
(Fig.~\ref{fig:SICM_STED1}a) and, subsequently, the cytoskeletal protein actin by 
confocal and STED microscopy (Fig.~\ref{fig:SICM_STED1}b). The area of the
sample imaged by both methods is marked in Fig.~\ref{fig:SICM_STED1}c (SICM: yellow 
rectangle, STED: gray rectangle). 



\begin{figure}	
  \centering
    \includegraphics[width=\textwidth]{gr/correlating/Hagemann2018-F4}
      \caption{
      \textbf{Correlative SICM and STED recording of a group of HeLa cells.} 
      (\textbf{Aa}) SICM recording of the HeLa cells, (\textbf{Ab}) slope representation of the
      SICM recording. Confocal (\textbf{Ba}) and STED (\textbf{Bb}) recording of the region marked 
      with a black dotted line in \textbf{Aa} and \textbf{Ab}. (\textbf{C}) Relative position of 
      the SICM (yellow rectangle) and the confocal/STED (gray rectangle) images against each 
      other. Magnification of the STED (\textbf{Da}), confocal (\textbf{Db}), SICM (\textbf{Dd}) 
      and slope (\textbf{De}) images. The region, that is magnified, is marked with a white 
      arrowhead in \textbf{Aa, Ab, Ac and Ad}. (\textbf{Df} and (\textbf{Dg}) show an overlay of 
      the STED with the SICM (\textbf{Df}) or the slope representation (\textbf{Dg}). 
      (\textbf{Dc}) Height profile along the white dotted line in \textbf{Dd}. \newline Reprinted 
      with permission from \emph{Philipp Hagemann*, Astrid Gesper*, Patrick Happel}, Correlative 
      stimulated emission depletion and scanning ion conductance microscopy, \emph{ACS Nano} 2018, 
      12, 5807-5815, DOI: 10.1021/acsnano.8b01731. Copyright 2020 American Chemical Society}
  \label{fig:SICM_STED1}
\end{figure}

Most interestingly, a protrusion (white arrowhead in
Fig.~\ref{fig:SICM_STED1}a, b) extending from one cell was captured by both
methods (magnified in Fig.~\ref{fig:SICM_STED1}d). The confocal recording
showed a blurred spot within the protrusion (Fig.~\ref{fig:SICM_STED1}db),
while the STED recording revealed that actin was arranged as a ring-shaped
structure around two central spots. The authors speculate that this structure
might be involved in the attachement or detachement of the cell from the cell
culture dish. However, since the recordings were performed on fixed cells, no
time-lapse data that could support this speculation exist.

% Bis hier

As mentioned in section~\ref{sec:correlating-sicm-and-fm}, SICM was also successfully combined 
with dSTORM (see section~\ref{sec:smlm}) \cite{Hennig2015a}. The difference to the 
SICM/STED microscopy correlation study was, that the SICM was in that study used for nanoinjecting 
fluorescently labeled molecules -- ATTO 655-phalloidin -- into living U2OS cells. 

% Hier jetzt noch was zu SICM/SNOM!

%%% Local Variables:
%%% mode: latex
%%% TeX-master: "../manuscript"
%%% End:
