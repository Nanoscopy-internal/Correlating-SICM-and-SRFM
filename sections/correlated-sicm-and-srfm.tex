\section{Correlating SICM and Super-Resolved Fluorescence Microscopy}
\label{sec:correlating-sicm-and-srfm}

The development of correlated
SICM and SRFM is still in its very early stages. The first combination of SRFM
and SICM has been the use of STORM to determine the number of fluorescent
molecules deposited via an SICM tip \cite{Hennig2015}. However, recently a
first proof-of-principle study combining STED and SICM has been published
\cite{Hagemann2018}.

In the latter, the topography of a group of fixed HeLa has been investigated
by SICM (Fig.~\ref{fig:SICM_STED1}a) and, subsequently, the distribution of
the cytoskeletal protein actin was determined by confocal and STED microscopy
(Fig.~\ref{fig:SICM_STED1}b) on a second instrument. The area of the sample
imaged by both methods is marked in Fig.~\ref{fig:SICM_STED1}c (SICM: yellow
rectangle, STED: gray rectangle).



\begin{figure}	
  \centering
    \includegraphics[width=\textwidth]{gr/correlating/Hagemann2018-F4}
      \caption{
      \textbf{Correlative SICM and STED recording of a group of HeLa cells.} 
      \textbf{a:} Height (aa) and slope representation (ab) of a SICM recording
      of HeLa cells. \textbf{b:} Confocal (ba) and STED (bb) recording of the
      region marked  
      with a black dotted line in a. \textbf{c:}  Relative position of 
      the SICM (yellow rectangle) and the confocal/STED (gray rectangle)
      images with respect to each 
      other. \textbf{d:} Magnification of the STED (da), confocal (db), SICM
      (dd) and slope (de) images. The magnified region is marked with a white 
      arrowhead in a and b. \textbf{df} and \textbf{dg} show an overlay of 
      the STED and the SICM data. 
      \textbf{Dc} shows the height profile along the white dotted line in dd.  Reprinted 
      with permission from Philipp Hagemann, Astrid Gesper, Patrick Happel: \emph{Correlative 
      stimulated emission depletion and scanning ion conductance microscopy.} ACS Nano 2018, 
      12, 5807-5815. Copyright 2018 American Chemical Society.}
  \label{fig:SICM_STED1}
\end{figure}

Most interestingly, a protrusion (white arrowhead in
Fig.~\ref{fig:SICM_STED1}a, b) extending from one cell was captured by both
methods (magnified in Fig.~\ref{fig:SICM_STED1}d). The confocal recording
showed a blurred spot within the protrusion (Fig.~\ref{fig:SICM_STED1}db),
while the STED recording revealed that actin was arranged as a ring-shaped
structure around two central spots (Fig.~\ref{fig:SICM_STED1}da). The authors
speculate that this structure might be involved in the attachement or
detachement of the cell from the cell culture dish. However, since the
recordings were performed on fixed cells, time-lapse data that supports this
speculation could not be obtained.

% Bis hier

% As mentioned in section~\ref{sec:correlating-sicm-and-fm}, SICM was also successfully combined 
% with dSTORM (see section~\ref{sec:smlm}) \cite{Hennig2015a}. The difference to the 
% SICM/STED microscopy correlation study was, that the SICM was in that study used for nanoinjecting 
% fluorescently labeled molecules -- ATTO 655-phalloidin -- into living U2OS cells. 

% Hier jetzt noch was zu SICM/SNOM!

%%% Local Variables:
%%% mode: latex
%%% TeX-master: "../manuscript"
%%% End:
