\section{Introduction}
\label{sec:introduction}
Many cellular processes like exo- and endocytosis, cell growth, division and
migration, to name only a few, require the rearrangement, growth or shrinkage
of the cellular membrane. Scanning Ion Conductance Microscopy (SICM)
\cite{Hansma1989} is a contact-free scanning probe technique which, if applied
to cellular specimen, provides information about the dynamics of the
topography of the cell membrane. However, the underlying dynamics of the
molecular machinery which orchestrates the cell membrane dynamics remain
unresolved by SICM investigations. The latter can be resolved by fluorescence
microscopy (FM), which allows tagging a protein of interest with a fluorescent
marker and observing its dynamics. Therefore, correlating fluorescence and
scanning ion conductance microscopy data might help to unravel the mechanisms
that underlie the cellular processes listed above that occur at the cellular
membrane.

\begin{figure}
  \sidecaption[t]
  \includegraphics{gr/intro/Fig_SICM-resolution}
  \caption{\textbf{Resolution of SICM.} \textbf{A} \textbf{B} Two
    consecutive SICM recordings of the equatorial segment of a living boar
    spermatozoon recorded 10 minutes apart with a scanning pipette with
    $r_\mathrm{i} \approx 6.75\,\text{nm}$}
  \label{fig:sicm-resolution}
\end{figure}
  
The resolution of SICM is determined by the inner opening radius
$r_\mathrm{i}$ of the scanning pipette and can be approximated as
$3r_\mathrm{i}$ (Fig.~\ref{fig:sicm-resolution}A)
\cite{rheinlaender:094905,Rheinlaender2015}.  Thus, the resolution of SICM is
only limited by the technical limits to produce pipettes with very small
openings. Pipettes with radii of approximately 6.75\,nm have been used in for
SICM imaging (Fig.~\ref{fig:sicm-resolution}B) \cite{Shevchuk2008}. In
contrast, the resolution in light microscopy is fundamentally limited by the
diffraction of light. As Ernst Abbe found out in 1873, a beam of light cannot
be focused to a spot smaller than the resolution limit $d_\text{min}$ \cite{Abbe1873}:
\begin{equation}
  d_\text{min} = \frac\lambda{2n\sin\alpha} = \frac\lambda{2\mathrm{NA}}\text{.}
  \label{eq:diffraction-limit}
\end{equation}
Here, $\lambda$ denotes the wavelength of the light used to generate the
image, $n$ the diffractive index of the medium between sample and objective
lense and $\alpha$ the half-cone opening angle of the objective lense. The
product $n\sin\alpha$ is called the numerical aperture $\mathrm{NA}$ of the
objective. The exact limit, of course, depends on $\lambda$ and the NA of the
objective, but as a rule of thumb, it is often stated that the resolution
limit is in the range of 200\,nm. This is approximately 14 times the size of
the smallest structures observed in the SICM recordings shown in
Fig.~\ref{fig:sicm-resolution}B, which were reported to be 14\,nm
\cite{Shevchuk2008}.

\begin{figure}
  \sidecaption
  \includegraphics{gr/intro/Fig_SICM-and-Light-resolution}%
  \caption{\textbf{Hypothetical fluorescence analysis of proteins in the boar
      spermatozoon.} \textbf{A:} Proteins A and B in the stable structures
    observed by SICM. \textbf{B:} Hypothetical positions of the fluorescent
    labels of proteins A and B. Scale bar: 200\,nm \textbf{C:} Computed
    fluorescence image.}
  \label{fig:sicm-and-light-resolution}
\end{figure}
In all stable structures in the recordings (highlighted by
the dashed lines) at least one group of three protrusions arranged in a line
are observable. If one assumes, without any biological support and just to be
able to exemplify the impact of the limited resolution of light microscopy,
that such a row is comprised of two different proteins, one at the first and
last position (magenta arrows in Fig.~\ref{fig:sicm-and-light-resolution}A),
and the other one on the central spot (green arrows in Fig.~\ref{fig:sicm-and-light-resolution}A), and one further assumes
that the two different proteins have been labelled by two different markers
(the positions of the markers are indicated by the magenta and green dots in Fig.~\ref{fig:sicm-and-light-resolution}B),
the resulting light microscopy images would result in      


% Since its invention, the light microscope is a widely used tool in
% biology. The development of the fluorescence and the confocal microscope
% \cite{Heimstaedt1911,Minsky1988} along with the improvement and development in
% staining techniques \cite{Coons1942,Tsien1998} made fluorescence microscopy
% (FM) a standard tool in the life sciences and medicine. With the new
% development of super-resolved fluorescence microscopy (SRFM)
% \cite{hell+wichmann,Betzig2006,Rust2006}, fluorescence microscopy entered the
% world beyond Abbe's diffraction limit \cite{Abbe1873}, and now allows to
% routinely investigate questions at a length scale below 100\,nm.

% SICM, as any scanning probe technique, provides information about the surface
% of the scanned sample. In contrast, FM provides information about the
% distribution of the labeled specimen. Gathering complementary information from
% the same sample, surface topography  

% \begin{equation}
%   \label{eq:abbe}
%   d = \frac{\lambda}{2 n \sin\alpha}
% \end{equation}

%%% Local Variables:
%%% mode: latex
%%% TeX-master: "../manuscript"
%%% End:
