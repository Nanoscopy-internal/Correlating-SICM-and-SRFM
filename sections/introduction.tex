\section{Introduction}
\label{sec:introduction}
Many cellular processes like exo- and endocytosis, cell growth, division and
migration, to name only a few, require the rearrangement, growth or shrinkage
of the cellular membrane. Scanning Ion Conductance Microscopy (SICM)
\cite{Hansma1989} is a contact-free scanning probe technique which, if applied
to cellular specimen, provides information about the dynamics of the
topography of the cell membrane. However, the underlying dynamics of the
molecular machinery which orchestrates the cell membrane dynamics remain
unresolved by SICM investigations. The latter can be resolved by fluorescence
microscopy (FM), which allows tagging a protein of interest with a fluorescent
marker and observing its dynamics. Therefore, correlating fluorescence and
scanning ion conductance microscopy data might help to unravel the mechanisms
that underlie the cellular processes listed above that occur at the cellular
membrane.

\begin{figure}\centering
  \includegraphics{gr/intro/Fig_SICM-resolution}
  \caption{\textbf{Resolution of SICM.}  \textbf{a:} Height profiles
    (plotted vertically) of modelled SICM recordings of two cilindrical
    particles (height and diameter: $r_\text{i}$) at varying
    distances. \textbf{b:} Apparent height of the particles (solid line) and of
    the gap between them (dahsed line) as determined by SICM. At distances
    $d\lessapprox3r_\text{i}$, no gap between the particles is detected.
    Reprinted from Johannes Rheinlaender, Tilman E. Schäffer: \emph{Image
      formation, resolution, and height measurement in scanning ion
      conductance microscopy}, Journal of Applied Physics 2009, 105(9), with
    the permission of AIP Publishing.  }
  \label{fig:sicm-resolution}
\end{figure}
  
The resolution of SICM is determined by the inner opening radius
$r_\mathrm{i}$ of the scanning pipette and can be approximated as
$3r_\mathrm{i}$ (Fig.~\ref{fig:sicm-resolution}a)
\cite{rheinlaender:094905,Rheinlaender2015}.  Thus, the resolution of SICM is
only limited by the technical limits to produce pipettes with very small
openings. Pipettes with radii of approximately 6.75\,nm have been used in for
SICM imaging \cite{Shevchuk2006}. In
contrast, the resolution in light microscopy is fundamentally limited by the
diffraction of light. As Ernst Abbe found out in 1873, a beam of light cannot
be focused to a spot smaller than the resolution limit $d_\text{min}$
\cite{Abbe1873}:
\begin{equation}
  d_\text{min} = \frac\lambda{2n\sin\alpha} = \frac\lambda{2\mathrm{NA}}\text{.}
  \label{eq:diffraction-limit}
\end{equation}
Here, $\lambda$ denotes the wavelength of the light used to generate the
image, $n$ the diffractive index of the medium between sample and objective
lense and $\alpha$ the half-cone opening angle of the objective lense. The
product $n\sin\alpha$ is called the numerical aperture $\mathrm{NA}$ of the
objective. The exact limit, of course, depends on $\lambda$ and the NA of the
objective, but as a rule of thumb, it is often stated that the resolution
limit is in the range of 200\,nm. This is approximately 14 times the size of
the smallest structures observed in the SICM recordings, which were reported to be 14\,nm
\cite{Shevchuk2006}.

\begin{figure}
  \centering
  \includegraphics{gr/intro/effect-of-resolution}%
  \caption{\textbf{Hypothetical fluorescence analysis of a protein forming
      membrane protrusions.} \textbf{a:} SICM recording of membrane protrusion
    and location of a putative protein within it. \textbf{b:} Computed
    diffraction limited fluorescence recording of the protein with a
    resolution of 250\,nm. \textbf{c:} Computed diffraction unlimited
    fluorescence recording of the protein with a resolution of 75\,nm.
    Reprinted with permission from Philipp Hagemann, Astrid Gesper, Patrick
    Happel: \emph{Correlative stimulated emission depletion and scanning ion
      conductance microscopy.} ACS Nano 2018, 12, 5807-5815. Copyright 2018
    American Chemical Society.}
  \label{fig:sicm-and-light-resolution}
\end{figure}

To exemplify the impact of the limited resolution of light microscopy, assume,
without any biological support, that a cellular protrusion as shown in
Fig.~\ref{fig:sicm-and-light-resolution}a is formed by a protein (black line
in Fig.~\ref{fig:sicm-and-light-resolution}a). A -- hypothetical --
diffraction limited recording of that protein would only show a blurred spot
(Fig.~\ref{fig:sicm-and-light-resolution}b), while a recording with an
improved resolution would clearly allow to see that the location of the
protein correlates with the position of the protrusion
(Fig.~\ref{fig:sicm-and-light-resolution}c). Note that the resolutions used to
compute the hypothetical fluorescent images correspond to resolutions
experimentally obtained on corresponding instruments \cite{Hagemann2018}. 

%%% Local Variables:
%%% mode: latex
%%% TeX-master: "../manuscript"
%%% End:
