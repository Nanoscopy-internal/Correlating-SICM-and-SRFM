\section{Potential Pitfalls}
\label{sec:pitfalls}
Combining SRFM and SICM in one setup is bound to bring up some potential
pitfalls, some of which are discussed in the following:

To avoid some problems from the beginning, both the SRFM and SICM setups
should provide spatial resolutions in the same range. Also, if possible, one
software should be used to control both setups to prevent delays in the
imaging process due to ... (?)

Some potential pitfalls can arise when aligning the SRFM and SICM
setups. There are two alignment options: The first one is aligning the pipette
tip and the laser beam only in the xy plane, e.g. for imaging fluorescent
proteins in the cytoskeleton and the cell membrane simultaneously. The second
one is additionally aligning the tip and beam in the z plane, e.g. to image
cell membrane and fluorescent particles or proteins within the membrane
simultaneously, as has been done in SSCM \cite{Gorelik2002a}\cite{Shevchuk2008}.

Especially the second option tracing the cell surface would require constant
focal plane adjustment of the optical setup. This might eventually lead to a
mismatch in the excitation and depletion beam superimposition.

It has to be put into consideration whether fine adjustment screws are enough
to align the pipette and the beam. It is apparently sufficient for aligning
SICM and confocal microscopes (??). In case it isn't accurate enough piezo
elements might have to be used. However, the usage of 3 piezo elements mounted
to the same platform, may it be the pipette holder or sample stage, is
reported to have caused cross-talk during sample positioning leading to image
artifacts in the range of 100 nm \cite{Shevchuk.2013}. Hence, it would be
better to uncouple xy piezos from the z piezo. Furthermore, z piezos come with
an increased risk of piezo drift. Therefore, especially the sample stage
should ideally avoid z piezo usage.

Both alignment options may lead to the pipette tip reflecting the beam and
thereby distorting the resulting recordings. In this case the pipette tip
would have to be retracted until it can't reflect the beam light anymore,
before a recording with the optical setup can be executed. Consequently, this
means the SRFM and the SICM recording can't be executed simultaneously but
successively.

While this might seem like a drawback extending the overall recording time, it
also solves another problem that comes with combining SRFM and SICM:
photo-bleaching and photo-damage of the sample. SRFM techniques like STED
microscopy come with a general risk of photo-bleaching and -damage as high
intensity lasers are used. Combining SRFM with SICM increases this risk,
as SICM is a much slower imaging technique. The pipette will not only have to
be moved in the xy plane but also in the z plane to avoid sample
contact. Hence, the imaging speed will mostly be dependent on the SICM
capillary speed especially in z direction. If the sample is illuminated by the
laser continuously during capillary approach and retraction, photo-bleaching
and -damage is likely to occur. Therefore, illuminating the sample right after
SICM surface detection, recording and pipette retraction for a short period of
time would not only result in reflection avoidance but also in less
photo-bleaching and -damage. In case, light reflection by the scanning
capillary doesn't occur, the sample could be illuminated right after surface
detection leading to a faster imaging speed of the combined setup.

The switching of the laser beams could either be done via mechanical
shutters. However, as mechanical shutters have a limited switching frequency,
it might be preferential to switch beams by switching akkusto-optical modulators.

Another problem stemming from capillary movement is the resonance of the
pipette resulting from it. Some groups have ensured resonance reduction by
using a v-groove mounting plate for the capillary instead of the conventional
patch clamp pipette holders \cite{Shevchuk.2013}.

 	

%%% Local Variables:
%%% mode: latex
%%% TeX-master: "../manuscript"
%%% End:
