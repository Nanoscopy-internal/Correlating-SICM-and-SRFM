\section{Potential Pitfalls}
\label{sec:pitfalls}

Combining SICM and FM or SRFM techniques leads to a variable quantity of 
correlating data depending on the overlap of data acquired by each technique.
Figure \ref{fig:CombinedMicroscopy} schematically illustrates implementations 
of combined microscopes and the respective resulting correlating data. 

\begin{figure}
  \sidecaption
  \includegraphics{gr/pitfalls/CombinedMicroscopy}%
  \caption{\textbf{Potential implementations of combined SICM and FM.} 
  			\textit{Left:} Schematic depiction of potential combined SICM 
  			and TIRF or standard wide-field microscopy (\textit{top}),
  			SICM and z-stacking (\textit{middle}) and SICM and surface
  			scanning (\textit{bottom}). 
  			\textit{Right:} Resulting correlating data of these combinations.}
  \label{fig:CombinedMicroscopy}
\end{figure}

A combination of SICM and wide-field microscopy techniques such as TIRF yields
two data sets, just as any other combination would. However, before these data
sets can be combined and examined for correlating data, it has to be ensured that 
a data point from one set is assigned a data point from the other set. As SICM 
can have significantly higher resolutions than wide-field microscopy, it is 
likely that one data point from the wide-field microscopy data set is assigned 
multiple data points from the SICM data set. To avoid this, the SICM pipette 
could be adjusted to yield a lower resolution comparable to the wide-field 
microscopy resolution. In case of a resolution mismatch the data sets have to be
aligned after imaging, as wide-field microscopy is not a scanning technique and
records all data at once instead of one another making it impossible to align the
microscopes and thus the data recording prior to imaging. Due to the resolution
mismatch this can prove difficult.

Once the data is aligned, it can be checked for correlations. As SICM traces the 
surface of the sample, it changes its position constantly in all 3 dimensions unless
the sample has a very flat surface and a change in z direction is not necessary.
In contrast, wide-field microscopy only records two dimensional data (in x and y
directions). This means, that data points from the two different sets only truly 
correlate, if the focus of the wide-field microscope is set to a focal plane (z)
in the same or nearly the same z position as the SICM. To achieve this, the SICM
should either not change its z position much, which is only reasonable for
relatively flat samples, or SICM data from different z planes has to be discarded.
Flat cell processes or extensions are an example for a structure, which could be 
investigated by a such a combination.

SICM and z-stacking FM techniques, when combined, yield more correlating data than
SICM and wide-field techniques as z-stacking microscopy as this FM technique
records data points in multiple z planes. After imaging these can be aligned with 
the SICM data set and non-correlating data can be separated from correlating data.
However, there is still a potential mismatch in resolution and the data has the 
processed after imaging to yield correlation. Yet, this implementation is suitable
for all, not only flat, samples.

Implementations of SICM and surface scanning microscopy automatically solve one
of the problems occuring when combining SICM with other FM techniques: Both 
microscopes are scanning techniques able to change their z position from one data
point to the next enabling an alignment before data recording. Therefore, all 
recorded data, if aligned correctly, can be assumed to be correlating.

Instead of having to worsen SICM resolution to match FM resolution, it might be 
reasonable to combine SICM and SRFM in one setup. Though, this implementation
comes with its own potential pitfalls, especially as due to the super resolution
of both techniques small deviations or disturbances are sufficient to ruin the 
alignment of the setup. Some of these potential pitfalls are discussed in the 
following:

To avoid some problems from the beginning, both the SRFM and SICM setups
should provide spatial resolutions in the same range. Also, if possible, one
software should be used to control both setups to prevent delays in the
imaging process due to transmitting information from one software to the other.

Some potential pitfalls can arise when aligning the SRFM and SICM
setups. There are two alignment options: The first one is aligning the pipette
tip and the laser beam only in the xy plane, e.g. for imaging fluorescent
proteins in the cytoskeleton and the cell membrane simultaneously. This is 
comparable, in terms of data recording, to the above mentioned combinations of
SICM and wide-field or z-stacking microscopy. 
The second one, surface scanning, is additionally aligning the tip and beam
in the z plane, e.g. to image cell membrane and fluorescent particles or proteins 
within the membrane simultaneously, as has been done in 
SSCM \cite{Gorelik2002a}\cite{Shevchuk2008}.

Especially the second option tracing the cell surface would require constant
focal plane adjustment of the optical setup. This might eventually lead to a
mismatch in the excitation and depletion beam superimposition.

It has to be put into consideration whether fine adjustment screws are enough
to align the pipette and the beam. It is apparently sufficient for aligning
SICM and confocal microscopes \cite{Gorelik2002a}\cite{Shevchuk2008}. 
In case it isn't accurate enough piezo elements might have to be used. 
However, the usage of 3 piezo elements mounted to the same platform, may it be
the pipette holder or sample stage, is reported to have caused cross-talk during
sample positioning leading to image artifacts in the range of 100 nm 
\cite{Shevchuk2013}. Hence, it would be better to uncouple xy piezos from 
the z piezo. Furthermore, z piezos come with an increased risk of piezo drift. 
Therefore, especially the sample stage should ideally avoid z piezo usage.

Both alignment options may lead to the pipette tip reflecting the beam and
thereby distorting the resulting recordings. In this case the pipette tip
would have to be retracted until it can't reflect the beam light anymore,
before a recording with the optical setup can be executed. Consequently, this
means the SRFM and the SICM recording can't be executed simultaneously but
successively.

While this might seem like a drawback extending the overall recording time, it
also solves another problem that comes with combining SRFM and SICM:
photo-bleaching and photo-damage of the sample. SRFM techniques like STED
microscopy come with a general risk of photo-bleaching and -damage as high
intensity lasers are used. Combining SRFM with SICM increases this risk,
as SICM is a much slower imaging technique. The pipette will not only have to
be moved in the xy plane but also in the z plane to avoid sample
contact. Hence, the imaging speed will mostly be dependent on the SICM
capillary speed especially in z direction. If the sample is illuminated by the
laser continuously during capillary approach and retraction, photo-bleaching
and -damage is likely to occur. Therefore, illuminating the sample right after
SICM surface detection, recording and pipette retraction for a short period of
time would not only result in reflection avoidance but also in less
photo-bleaching and -damage. In case, light reflection by the scanning
capillary doesn't occur, the sample could be illuminated right after surface
detection leading to a faster imaging speed of the combined setup.

The switching of the laser beams could either be done via mechanical
shutters. However, as mechanical shutters have a limited switching frequency,
it might be preferential to switch beams by switching akkusto-optical modulators.

Another problem stemming from capillary movement is the resonance of the
pipette resulting from it. Some groups have ensured resonance reduction by
using a v-groove mounting plate for the capillary instead of the conventional
patch clamp pipette holders \cite{Shevchuk2013}.

 	

%%% Local Variables:
%%% mode: latex
%%% TeX-master: "../manuscript"
%%% End:
