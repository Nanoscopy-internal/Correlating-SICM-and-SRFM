\section{Outlook}
\label{sec:pitfalls}

In the previous sections, we have reviewed the combination of SICM and FM and
have shown that this combination allows to gather new insights into the
physiology of living cells. However, despite the mismatch in resolution, most
of these studies combine SICM and diffraction limited FM. Even other scanning
probe techniques such as atomic force microscopy, which has a much larger user
base than SICM, have been combined with SRFM only in a few studies
\cite{Harke2012,Chacko2013,Chacko2013b,Odermatt2015,Curry2017,Hirvonen2018}.

We can only speculate about the reasons for this. In its most simple
implementation, SICM and SRFM data could be recorded on two different
instruments subsequently \cite{Hagemann2018}. The major problem with this
approach is finding the same region of the sample in both instruments. Here,
strategies from correlating FM and electron microscopy (recently reviewed for
example in \cite{Begemann2016}) might be adapted. A second drawback when using
TIRF-based methods such as PALM or STORM is the limited penetration depth of
the excitation beam, which only allows recording fluorescence data close to
the interface between sample and cell culture dish. However, approaches
allowing 3D imaging by localisation microscopy exist
\cite{Huang2008a,Huang2008} and might allow to circumvent this limit.  

% Instead of having to worsen SICM resolution to match FM resolution, it might be 
% reasonable to combine SICM and SRFM in one setup. Though, this implementation
% comes with its own potential pitfalls, especially as due to the super resolution
% of both techniques small deviations or disturbances are sufficient to ruin the 
% alignment of the setup. Some of these potential pitfalls are discussed in the 
% following:

% To avoid some problems from the beginning, both the SRFM and SICM setups
% should provide spatial resolutions in the same range. Also, if possible, one
% software should be used to control both setups to prevent delays in the
% imaging process due to transmitting information from one software to the other.

In the case of combining SICM with a scanning technique such as STED, there
are two alignment options: The first one is aligning the pipette tip and the
laser beam only in the $x,y$-plane. This is comparable, in terms of data
recording, to the above mentioned combinations of SICM and wide-field or
confocal microscopy. The second one, surface scanning, requires the
additional alignment of the tip and, in case of STED, the two beams in the
$z$-plane.

Especially the second option would require repeated
focal plane adjustment of the optical setup. In turn, this requires meticulous
alignment of the excitation and depletion beam to retain the super-imposition
of the two beams during the adjustment of the focal plane. To minimize the
pixels that require an adjustment of the focal plane, the adjustment has been
only applied if a $z$-range of more than one micrometer was exceeded
\cite{Novak2014}. If STED recordings with increased vertical resolution are
intended, the acceptable range for not adjusting the focal plane decreases,
which in turn might lead to longer and less stable recordings. 

Due to the increased horizontal resolution, even the $x,y$-alignment of SICM
pipette and the laser beams becomes more challenging since the tolerable
offset between the two decreases. While it has been reported that fine
adjustment screws are sufficient to align the pipette and the beam in the
combination of SICM and confocal microscopes \cite{Gorelik2002a,
  Shevchuk2008,Novak2014}, additional piezo elements might be required for the
alignment of STED and SICM.

If the depletion laser and the pipette tip are aligned in all three
dimensions, the pipette tip might reflect the depletion beam. Since the
intensity of the beam is large to achieve a high resolution, the reflected
light might not be completely filtered by the filters in the emission beam
path. In this case, the pipette would have to be retracted first before the
recording of the fluorescence data could be performed.

This, in turn, requires a precise control of the data acquisition by the two
imaging methods, which can be only obtained by a designated controlling soft-
and hardware.

In summary, the development of combined SICM and SRFM instruments has just
begun, and several technical hurdles have to be overcome during the
further development. However, there are no fundamental contradictions that
would hinder the combination of SICM and SRFM. Therefore, we expect this field
of research to grow in the near future.     

\section{Acknowledgements}
The authors acknowledge support from the Volkswagen
Foundation (grant 88 390) and the German Research Foundation (grant
411517989). 
%%% Local Variables:
%%% mode: latex
%%% TeX-master: "../manuscript"
%%% End:
