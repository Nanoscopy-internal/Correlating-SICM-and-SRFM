% This is the main manuscript file. The preamble is suggested by Springer and
% should not be changed.
%
% 
\documentclass[graybox]{svmult}

% choose options for [] as required from the list
% in the Reference Guide

\usepackage{type1cm}        % activate if the above 3 fonts are
                            % not available on your system
%
\usepackage{makeidx}         % allows index generation
\usepackage{graphicx}        % standard LaTeX graphics tool
                             % when including figure files
\usepackage{multicol}        % used for the two-column index
\usepackage[bottom]{footmisc}% places footnotes at page bottom


\usepackage{newtxtext}       % 
\usepackage{newtxmath}       % selects Times Roman as basic font

% see the list of further useful packages
% in the Reference Guide

\makeindex             % used for the subject index
                       % please use the style svind.ist with
                       % your makeindex program

%%%%%%%%%%%%%%%%%%%%%%%%%%%%%%%%%%%%%%%%%%%%%%%%%%%%%%%%%%%%%%%%%%%%%%%%%%%%%%%%%%%%%%%%%


\newcommand\todo[1]{\marginpar{\small\itshape\textcolor{red}{#1}}}
\begin{document}

\title*{Correlating Scanning Ion Conductance and Super-Resolved Fluorescence Microscopy}
\titlerunning{Correlating SICM and SRFM}

\author{Patrick Happel \and Astrid Gesper \and Annika Haak}
\authorrunning{Happel et al.}
% Use \authorrunning{Short Title} for an abbreviated version of
% your contribution title if the original one is too long

\institute{%
  Annika Haak\and
  Astrid Gesper \and
  Patrick Happel \at
  Nanoscopy Group,
  RUBION, Ruhr-Universität Bochum,
  Universtätsstraße 150, D-44801 Bochum, Germany
  \email{patrick.happel@rub.de}
}
  
\maketitle


%\abstract*{Each chapter should be preceded by an abstract (no more than 200 words) that summarizes the content. The abstract will appear \textit{online} at \url{www.SpringerLink.com} and be available with unrestricted access. This allows unregistered users to read the abstract as a teaser for the complete chapter.
%Please use the 'starred' version of the \texttt{abstract} command for typesetting the text of the online abstracts (cf. source file of this chapter template \texttt{abstract}) and include them with the source files of your manuscript. Use the plain \texttt{abstract} command if the abstract is also to appear in the printed version of the book.}

\abstract{%
  The development of diffraction unlimited fluorescence microscopy has paved
  the way for a detailed understanding of cellular dynamics since live-cell
  investigations with a resolution below 100\,nm have become
  available, allowing to investigate 
  labeled specimen such as proteins and their distribution within a cell at a
  new level of detail. However, the place of the interaction between the
  cell's internal processes and external signals, the cell membrane, cannot be
  investigated by fluorescence microscopy since inserting fluorescent dyes
  alters the physico-chemical properties of the cell membrane.
%
  Investiganting the cell membrane is a unique strength of scanning ion conductance
  microscopes (SICM),
  which use a fine tip to determine the topography of the investigated sample
  and thus allows to investigate the cell membrane without labelling.    
%  
  Since both types of microscopes can be applied to living cells and provide a
  resolution below the diffraction limit, correlating data from the two
  seems a promising approach to further the understanding of
  the organisation of the cell plasma membrane and of the mechanisms involved
  in the transport processes across it.
%  
  Here, we briefly introduce the different approaches to Super-Resolved
  Fluorescence Microscopy (SRFM) and review the applications of correlating SICM
  and diffraction-limited as well as diffraction-unlimited fluorescence
  microscopy. Lastly, we discuss potential implementations of combined
  SRFM/SICM instruments and the potential pitfalls that may arise during the
  development of such, hitherto putative, instruments.
}

\tableofcontents

\section{Introduction}
\label{sec:introduction}
Many cellular processes like exo- and endocytosis, cell growth, division and
migration, to name only a few, require the rearrangement, growth or shrinkage
of the cellular membrane. Scanning Ion Conductance Microscopy (SICM)
\cite{Hansma1989} is a contact-free scanning probe technique which, if applied
to cellular specimen, provides information about the dynamics of the
topography of the cell membrane. However, the underlying dynamics of the
molecular machinery which orchestrates the cell membrane dynamics remain
unresolved by SICM investigations. The latter can be resolved by fluorescence
microscopy (FM), which allows tagging a protein of interest with a fluorescent
marker and observing its dynamics. Therefore, correlating fluorescence and
scanning ion conductance microscopy data might help to unravel the mechanisms
that underlie the cellular processes listed above that occur at the cellular
membrane.

\begin{figure}\centering
  \includegraphics{gr/intro/Fig_SICM-resolution}
  \caption{\textbf{Resolution of SICM.}  \textbf{a:} Height profiles
    (plotted vertically) of modelled SICM recordings of two cilindrical
    particles (height and diameter: $r_\text{i}$) at varying
    distances. \textbf{b:} Apparent height of the particles (solid line) and of
    the gap between them (dahsed line) as determined by SICM. At distances
    $d\lessapprox3r_\text{i}$, no gap between the particles is detected.
    Reprinted from Johannes Rheinlaender, Tilman E. Schäffer: \emph{Image
      formation, resolution, and height measurement in scanning ion
      conductance microscopy}, Journal of Applied Physics 2009, 105(9), with
    the permission of AIP Publishing.  }
  \label{fig:sicm-resolution}
\end{figure}
  
The resolution of SICM is determined by the inner opening radius
$r_\mathrm{i}$ of the scanning pipette and can be approximated as
$3r_\mathrm{i}$ (Fig.~\ref{fig:sicm-resolution}a)
\cite{rheinlaender:094905,Rheinlaender2015}.  Thus, the resolution of SICM is
only limited by the technical limits to produce pipettes with very small
openings. Pipettes with radii of approximately 6.75\,nm have been used in for
SICM imaging \cite{Shevchuk2006}. In
contrast, the resolution in light microscopy is fundamentally limited by the
diffraction of light. As Ernst Abbe found out in 1873, a beam of light cannot
be focused to a spot smaller than the resolution limit $d_\text{min}$
\cite{Abbe1873}:
\begin{equation}
  d_\text{min} = \frac\lambda{2n\sin\alpha} = \frac\lambda{2\mathrm{NA}}\text{.}
  \label{eq:diffraction-limit}
\end{equation}
Here, $\lambda$ denotes the wavelength of the light used to generate the
image, $n$ the diffractive index of the medium between sample and objective
lense and $\alpha$ the half-cone opening angle of the objective lense. The
product $n\sin\alpha$ is called the numerical aperture $\mathrm{NA}$ of the
objective. The exact limit, of course, depends on $\lambda$ and the NA of the
objective, but as a rule of thumb, it is often stated that the resolution
limit is in the range of 200\,nm. This is approximately 14 times the size of
the smallest structures observed in the SICM recordings, which were reported to be 14\,nm
\cite{Shevchuk2006}.

\begin{figure}
  \centering
  \includegraphics{gr/intro/effect-of-resolution}%
  \caption{\textbf{Hypothetical fluorescence analysis of a protein forming
      membrane protrusions.} \textbf{a:} SICM recording of membrane protrusion
    and location of a putative protein within it. \textbf{b:} Computed
    diffraction limited fluorescence recording of the protein with a
    resolution of 250\,nm. \textbf{c:} Computed diffraction unlimited
    fluorescence recording of the protein with a resolution of 75\,nm.
    Reprinted with permission from Philipp Hagemann, Astrid Gesper, Patrick
    Happel: \emph{Correlative stimulated emission depletion and scanning ion
      conductance microscopy.} ACS Nano 2018, 12, 5807-5815. Copyright 2018
    American Chemical Society.}
  \label{fig:sicm-and-light-resolution}
\end{figure}

To exemplify the impact of the limited resolution of light microscopy, assume,
without any biological support, that a cellular protrusion as shown in
Fig.~\ref{fig:sicm-and-light-resolution}a is formed by a protein (black line
in Fig.~\ref{fig:sicm-and-light-resolution}a). A -- hypothetical --
diffraction limited recording of that protein would only show a blurred spot
(Fig.~\ref{fig:sicm-and-light-resolution}b), while a recording with an
improved resolution would clearly allow to see that the location of the
protein correlates with the position of the protrusion
(Fig.~\ref{fig:sicm-and-light-resolution}c). Note that the resolutions used to
compute the hypothetical fluorescent images correspond to resolutions
experimentally obtained on corresponding instruments \cite{Hagemann2018}. 

%%% Local Variables:
%%% mode: latex
%%% TeX-master: "../manuscript"
%%% End:


\subsection{Fluorescence Microscopy}
\label{sec:fm}
In order to understand how super-resolved fluorescence microscopy (SRFM)
circumvents the diffraction limit, it is helpful to recapitulate the operating
principle of fluorescence microscopy (FM). Fluorescence is the process of the
absorption of a photon by a molecule, the fluorophore, followed by the
spontaneous emission of a second photon. The transitions between the energetic
states of the fluorophore are depicted in
Fig.~\ref{fig:jablonski-fluorescence}a. Due to the non-radiative loss of
energy during the involved vibrational decays (black arrows in
Fig.~\ref{fig:jablonski-fluorescence}a), the energy $h\nu_\text{em}$ of the
emitted photon is smaller than the energy of the absorbed photon
$h\nu_\text{ex}$ (here, $h$ denotes the Planck constant). Since the frequency $\nu$
of a photon is given by $c\lambda^{-1}$ (with $c$: speed of light, $\lambda$:
wavelength of the photon), the wavelength $\lambda_\text{em}$ of the emitted
photon is larger than the wavelength $\lambda_\text{ex}$ of the absorbed
photon.      

\begin{figure}
%  \sidecaption
  \centering
  \includegraphics{gr/intro/Jablonski-fluorescence}
  
  \caption{%
    \textbf{Principle of fluorescence microscopy.}  \textbf{a:} Jablonski
    diagram of fluorescence. A photon of energy $h\nu_\text{ex}$ gets absorbed
    and excites the molecule from its electronic ground state $S_0$ to a
    vibrational excited state of the electronic state $S_1$ (green
    arrow). After radiation-less relaxation (black arrow) into the vibrational
    ground state of $S_1$, the molecule returns to one of the vibrational
    states of $S_0$ (red arrow) by emitting a photon of energy
    $h\nu_\text{em}$, after which it relaxes to the lowest vibrational state
    of $S_0$.  \textbf{b:} Typical absorbance (green) and emission (red)
    spectra of a fluorescent dye molecule. Green and red rectangles indicate
    the wavelength range used for excitation (exc) and detection in FM.  }
  \label{fig:jablonski-fluorescence}
\end{figure}

Typical absorbance/emission spectra of fluorophores used as dyes in FM are
shown in Fig.~\ref{fig:jablonski-fluorescence}b. In FM, a single wavelength or
a wavelength range (green rectangle in Fig.~\ref{fig:jablonski-fluorescence}b) 
suitable to excite the fluorophore used is chosen -- by using a suitable laser
or LED or by selecting the wavelength range via a band-pass filter from a
white light source -- and send through the objective lens of the microscope to
the sample. In most FMs (the so called epifluorescence microscopes), the
emission is collected via the same objective lens. Since the emission is
red-shifted with respect to the excitation wavelength, the emitted photons
(wavelength range depicted by red rectangle in
Fig.~\ref{fig:jablonski-fluorescence}b) can be separated from the ones used to
excite the fluorophores by a dichroic mirror.

\begin{figure}
  \includegraphics{gr/intro/FM-setups}
  \caption{\textbf{Beam paths and major components of fluorescent
      microscopes.} \textbf{a:} Excitation (green) and emission (red) beam
    paths of a wide-field microscope. \textbf{b:} Excitation (green) and
    emission beam paths (red) of a confocal microscope.  \textbf{c:} Emission
    beam path (filters omitted for clarity) of a confocal microscope showing
    how out-of-focus emission (dashed and dotted lines) is suppressed by the
    confocal pinhole.}
  \label{fig:FM-setups}
\end{figure}

The two most common implementations of FM used in the life sciences are the
wide-field and the confocal microscope. Both use a dichroic mirror to separate
excitation and emission, but their implementations differ in the way how the
sample is excited and how the emission is recorded. The beam paths and the
major components of a wide-field and a confocal FM are depicted in
Fig.~\ref{fig:FM-setups}. In a wide-field FM (Fig.~\ref{fig:FM-setups}a), the
entire sample is illuminated at the same time, and all fluorophores in the sample
emit at the same time (yellow circles in the right inset in
Fig.~\ref{fig:FM-setups}a). The emission of the fluorophores (solid red shape
in Fig.~\ref{fig:FM-setups}a, only five beam paths are shown for clarity) is
collected by the objective lens, passes the dichroic mirror and is further
cleared by a band-pass filter. The emission then is focused by the tube lens
onto the camera, which records an image of the emission of all fluorophores at
the same time.

Each fluorophore in the sample generates a diffraction limited spot on the
camera, and spots which are not separated in space by at least the diffraction limit
cannot be distinguished as single spots. Thus, the wavelength which determines
the resolution in a wide-field microscope is the wavelength $\lambda_\text{em}$
of the emission.

The major disadvantage of a wide-field microscope is that the sample is not
only illuminated in the focal plane of the objective lens, but also
fluorophores above and below the focal plane are excited and their
emission is collected (not shown for clarity). Since they are not located in
the focal plane of the objective lens, their emission does not result in a
sharp spot on the camera, but instead appears as a blurred background.    

This led to the invention of the confocal microscope
\cite{Heimstaedt1911,Minsky1988}. The beam path of a confocal microscope is
shown in Fig.~\ref{fig:FM-setups}b. Here, the excitation light (green solid
shape) is focused onto the sample by the objective lens and only the
fluorophores within the resulting diffraction limited spot are excited. Thus, the
wavelength which determines the resolution of a confocal microscope is the
excitation wavelength $\lambda_\text{ex}$.  In contrast to the wide-field
microscope, the emission of the fluorophores is focused onto a pinhole which
is located in front of a single detector (left inset in
Fig.~\ref{fig:FM-setups}b). Since this setup does not allow imaging the entire
sample at once, the sample has to be scanned with respect to the excitation
light beam and the image is collected pixel by
pixel. Figure~\ref{fig:FM-setups}c illustrates the major effect of the pinhole:
Emission from out-of-focus planes (dotted and dashed lines) is blocked by the
pinhole since it is located in the same focal plane as the sample (thus the
name \emph{confocal}). 

For a potential combination with SICM, the most important aspect of the two
techniques is the way the fluorescence is recorded. Wide-field FM uses a 
camera to record the entire sample at once, whereas confocal FM is a scanning
technique which records the single pixels subsequently, either by moving the
sample through the beam or vice versa.

\subsection{Circumventing the diffraction limit}
\label{sec:circumvent-diffraction-limit}
The general approach to circumvent the diffraction limit is to reduce the
fluorophores within a diffraction limited spot which contribute to the
recorded signal, that is, to turn the fluorophores into an ``off''-state. Only
the fluorophores which contribute to the signal remain in the ``on''-state.
However, a few more approaches exist, for example 4$\pi$-microscopy
\cite{Hell1994} or structured illumination microscopy (SIM)
\cite{Guerra1995}. The first uses two objectives on opposite sides of the
sample, which renders it impossible to be combined with SICM since the space
required for the scanning pipette is already occupied by the second
objective. In contrast, SIM is implemented on a wide-field microscope and
could thus be combined with SICM. Therefore, combining SIM and SICM is similar
to combining SICM with other wide-field based techniques (see
section~\ref{sec:smlm}). Since the gain in resolution is low compared with the
approaches explained in sections~\ref{sec:smlm}
and~\ref{sec:deterministic-approaches}, we refrain from explaining SIM in
detail here.


\subsubsection{Stochastic approaches}
\label{sec:smlm}
The stochastic approaches to SRFM use dye molecules which can be switched
between an inactivated (``off'') and an activated (``on'') state by light of a
specific wavelength. If only a sparse subset of molecules is activated, such
that, on average, only one dye molecule per diffraction limited spot emits,
the position of the emitting molecule can be localised with much better
precision than the diffraction limit. If this is repeated until all
fluorophores have emitted, plotting the positions instead of the emission of
all fluorophores yields an image with resolution beyond the diffraction
limit (localisation microscopy). The first techniques using this principle
were photo-activated localisation microscopy (PALM) \cite{Betzig2006} (using
the Eos fluorescent protein \cite{Wiedenmann2004} as marker), fluorescence
photo-activation localisation microscopy (FPALM) \cite{Hess2006} (using
photo-activatable GFP as marker), and stochastical
optical reconstruction microscopy (STORM) \cite{Rust2006} (using Cy5 and Cy3
dyes).

\begin{figure}
  \centering
  \includegraphics{gr/intro/localisation-mic}
  \caption{%
    \textbf{Principle of PALM.}
    \textbf{a:} Initially, all fluorophores are in the inactivated state (grey
    dots).
    \textbf{b:} A small subset of the fluorophores is activated (cyan
    background, white dots) and subsequently excited (green background, yellow
    dots) until the fluorophores bleach (black dots). The emission of the
    fluorophores is recorded. This series is repeated until all fluorophores
    have emitted.
    \textbf{c:} The positions of the fluorophores in the recordings are
    determined and plotted as a Gaussian distribution with a standard
    deviation equal to the uncertainty of the localisation and the single
    images are combined to a single image with improved resolution.
  }
  \label{fig:lm}
\end{figure}

As an example, the principle of PALM is illustrated in Fig.~\ref{fig:lm}. Initially,
all fluorophores within the sample are inactivated (``off''-state, grey dots
in Fig.~\ref{fig:lm}a). Next, light suitable to switch the fluorophores (cyan
background, Fig.~\ref{fig:lm}b) from the inactivated to the activated (``on'')
state (white dots) is applied. The intensity of the activation light is
selected such that only a small subset of the inactivated molecules is
activated, at maximum one molecule per area of the diffraction limited
spot. The activated molecules are excited (green background, yellow dots) and
their emission is recorded. The excitation light is applied as long as all
excited molecules have bleached (black dots). The series of activation,
excitation and read-out is repeated until all fluorophores have been excited
and bleached, which results in a set of images containing diffraction limited
emission spots of all fluorophores in the sample.

In a post-processing step (Fig.~\ref{fig:lm}c), the positions of the
fluorophores are determined by fitting (an approximation of) the instrument's
point spread function to the data. The positions are plotted as Gaussian
distributions with a standard deviation equal to the uncertainty of the
determination of the position by the fit, finally yielding an image with an
improved resolution (Fig.~\ref{fig:lm}c, right).

The resolution provided by this approach scales with the number of photons $N$
that are detected and theoretically is given by $\sigma_\text{PSF}/\sqrt{N}$
(with $\sigma_\text{PSF}$ denoting the standard deviation of the point spread
function of the microscope, which is in the range of 100\,nm). However, in
practice the resolution is in the range of 20\,nm \cite{Betzig2006,Rust2006}
since the theoretical resolution cannot be reached due to background, noise
and other factors. A more detailed description of the resolution enhancement
by localisation is provided by \cite{Mortensen2010}. 

This brief explanation is far from being exhaustive and does not consider the
differences in the implementations of the various techniques. An in-depth
overview is provided by several reviews of the techniques
\cite{Patterson2010,Sengupta2014,Liu2015,Sauer2017}, suitable dyes
\cite{Li2018a} as well as by detailed imaging protocols
\cite{Gould2009,Schermelleh2010,Linde2011}. However, one aspect of the
implementation which is important for the potential combination with SICM is
the illumination scheme used by these methods. Since wide-field illumination
is required to record the emission of the activated fluorophores, out-of-focus
emission occurs, which increases the background in the recordings and affects
the resolution. To avoid this, a total internal reflection (TIRF) illumination
scheme \cite{AMBROSE1956,Axelrod1981} is used, which limits the excitation to
the layer close to the interface of the microscopy slide and the sample.   


\subsubsection{Deterministic approaches}
\label{sec:deterministic-approaches}
In contrast to the stochastic approaches, which switch the fluorophores
randomly in space, the deterministic approaches use a single-spot illumination
and switch the fluorophores in the periphery of the excitation spot. The
underlying principle is called reversible saturable optical fluorescence
transitions (RESOLFT). Methods that use this principle are stimulated emission
depletion (STED) microscopy \cite{hell+wichmann,Klar2000}, ground state
depletion (GSD) microscopy \cite{Hell1995,Bretschneider2007} and RESOLFT with
photo-switchable proteins \cite{Hofmann2005}. The latter is often referred to
as RESOLFT microscopy, although STED and GSD use the RESOLFT principle, too. 

\begin{figure}
  \centering
  \includegraphics{gr/intro/direct-SR-mic}
  \caption{\textbf{Principle of STED microscopy.} \textbf{a:} Shape of
    excitation and ring-shaped depletion beam. \textbf{b:} The two beams are
    superimposed and the sample is scanned. Fluorophores in the center are
    excited and emit spontaneously (yellow dot), while the depletion beam
    induces stimulated emission of the fluorophores in the periphery (black
    dots). \textbf{c:} (left) Jablonski diagram of stimulated emission (SE);
    (center) Line profile of the intensity of the depletion beam (red) and
    probability (prob) to induce stimulated emission (black); (right)
    wavelengths of excitation (ex), detection (det) and stimulated emission
    (SE) superimposed on the absorbance/emission spectrum of a fluorescent
    dye.}
  \label{fig:STED-principle}
\end{figure}

As an example for RESOLFT, the principle of STED microscopy is sketched in
Fig.~\ref{fig:STED-principle}. The excitation spot
(Fig.~\ref{fig:STED-principle}a, top) is superimposed with a second, ring- or
doughnut-shaped spot (Fig.~\ref{fig:STED-principle}a, bottom). The
fluorophores are excited by the excitation light, but only those located in
the center of the minimum of the depletion beam are allowed to emit
spontaneously (yellow dot). The fluorophores at the periphery are de-excited
by the depletion beam and undergo stimulated emission (black dots). The
Jablonski diagram of stimulated emission is shown in
Fig.~\ref{fig:STED-principle}c (left). An incoming photon of the depletion
beam interacts with the excited fluorophore and causes it to de-excite to the
ground state and to emit a second photon of the same energy as the incoming
photon. Due to the non-linear relation between intensity and the probability
to induce stimulated emission and the shape of the depletion beam
(Fig.~\ref{fig:STED-principle}c, center), the size of the origin of
spontaneous emission in the center of the depletion beam is decreased. The
wavelength of the depletion beam and thus the wavelength of stimulated
emission is selected such that it is outside the detection window of the
microscope (Fig.~\ref{fig:STED-principle}c, right), which de-facto switches
fluorophores that undergo stimulated emission to an invisible ``off''-state.

In summary, this directly yields a recording of the fluorophores with a
resolution beyond the diffraction limit (Fig.~\ref{fig:STED-principle}d). The
resolution provided by STED microscopy depends on the intensity $I$ of the
depletion beam and is given by
\begin{equation}
  d_\text{min, STED} = \frac\lambda{2\mathrm{NA}\sqrt{I/I_\text{sat}}}\text,
  \label{eq:resolution-STED}
\end{equation}
which is a modification of Abbe's original equation
(Eq.~\ref{eq:diffraction-limit}). Here, $I_\text{sat}$ is the intensity
required to induce stimulated emission in half of the fluorophores. The best resolution (approximately 2\,nm) obtained by STED microscopy has
been recorded on samples containing fluorescent color centers in diamond
crystal \cite{Wildanger2012}.

Note that, by applying a depletion beam of a
different shape, STED microscopy allows to improve the resolution along the
optical axis \cite{Harke2008, Wildanger2009}. A large number of reviews articles exist
which summarize the use of STED (or in general, RESOLFT) microscopy in different
fields in the life sciences, much too much to be listed here. We would like to
refer the reader to exhaustive reviews, which, besides
reviewing various applications, detail technical aspects of STED microscopy
\cite{Turkowyd2016,Blom2017} as well as to a review which contains some
anecdotal sections on how STED was developed \cite{Sahl2019}.


% The first concept to circumvent the diffraction limit was published in 1994
% and realised in 2000 \cite{hell+wichmann,Klar2000}


% Ernst Abbe found in 1873 \cite{Abbe1873} that the resolution $d$ of
% conventional light microscopes is limited by diffaction to 
% \begin{equation}
%   d_\text{min} = \frac{\lambda}{2\mathrm{NA}} = \frac\lambda{2n\sin\alpha}\text.
%   \label{eq:abbe}
% \end{equation}
%  One consequence of the diffraction limit is
% that a beam of light cannot be focused to a spot smaller than $d_\text{min}$.

% The place in the micropscope where the diffraction limited spot limits the
% resolution of a microscopic recording depends on the implementation of the
% microscope.  

% \begin{figure}
% %  \includegraphics{gr/introduction/}
% \end{figure}

% \subsection{General approach}

% \subsubsection{Different approaches}
% While the majority of SRFM techniques follows the approach outlined above,
% other techniques exist that use different approaches.  




% \subsection{Scanning techniques}

%%% Local Variables:
%%% mode: latex
%%% TeX-master: "../manuscript"
%%% End:

\section{Correlating SICM and Fluorescence Microscopy}
\label{sec:correlating-sicm-and-fm}

% Assume a figure here which shows the different degrees of correlation
\begin{figure}
  \includegraphics[width=\textwidth]{gr/correlating/sicm+fm}
  \caption{%
    \textbf{Combining SICM and wide-field fluorescence microscopy.}
    \textbf{a:}
    \textbf{b:} FM image of presynaptic nerve terminals stained with
    FM1-43. \textbf{c:} Overlay of the FM-image and two SICM recordings of the
    same region. \textbf{d:} 3D-representation of the lower right SICM
    recording from c.
    \textbf{b--d:} Reprinted with permission from ref \cite{Scheenen2015}.
  }
  \label{fig:sicm+fm}
\end{figure}

Correlative SICM and FM, despite the mismatch in resolution, allows to gather
information which cannot be obtained by using a single technique
alone. However, the degree of correlation as well as the technical
implementation differs, depending on the FM technique combined with SICM, as
we will detail in the following sections.

Although recording SICM and FM data independently on different samples under
the same conditions can support conclusions that could not be drawn from data
from a single technique alone \cite{Gesper2017,Lee2013,Lyon2009}, we will
focus on applications were SICM and FM data has been recorded from the same
sample.


\subsection{Combining SICM and wide-field FM}
\label{sec:SICM+widefield}
Combining SICM and wide-field FM allows to image the same region of the sample
with both techniques (Fig.~\ref{fig:sicm+fm}a). Due to the different fields of
view, the different pixel sizes and the different times required to record one
image, the two data sets are not recorded synchronously and thus cannot by
correlated pixel-by-pixel. Technically, the combination is simple, since
the SICM only has to be mounted onto an inverted optical FM and both
instruments can be controlled independently by their own software.

So far, the combination of SICM and wide-field FM has been used to find a
region of interest such as an area containing active synapses in a culture of
a neuronal network \cite{Scheenen2015} (Fig.~\ref{fig:sicm+fm}b--d) or to
identify the neuron which had been investigated by SICM-guided patch-clamp of
synaptic boutons \cite{Novak2013}.

Furthermore, it has been used to first identify primary cilia in a cell
culture and subsequently use SICM to reveal whether the selected cilium was
extending into the extracellular space or was embedded in the cell, an
information which could not be retrieved by using FM alone \cite{Zhou2018}.


\begin{figure}
  \includegraphics[width=\textwidth]{gr/correlating/Nikolaev2010-F2}
  \caption{
    \textbf{a:}\textbf{b:}\textbf{c:}\textbf{d:}\textbf{e:}\textbf{f:}
  }
  \label{fig:nikolaev2010}
\end{figure}
Cardio-myocytes exhibit a unique topography with transverse (T-)tubules and
crests (Fig.~\ref{fig:nikolaev2010}a, b). By first recording the topography
and subsequently moving the SICM pipette to either a T-tubule or a crest,
followed by the local application of agonists of either the $\beta_1$- or
$\beta_2$-adrenergic receptor (AR) and recording the intensity ratio of cyan
and yellow fluorescent protein, coupled to a Förster resonance energy transfer
(FRET) based reporter of cyclic AMP (Fig.~\ref{fig:nikolaev2010}c--f), it was
shown that the distribution of the $\beta_2$-AR was altered in failing
cardio-myocytes \cite{Nikolaev2010}. Furthermore, this approach was used to
investigate the effect of partial mechanical unloading, showing that it
reduces the cyclic AMP response of $\beta_2$-ARs at the T-tubules
\cite{Wright2018}. To investigate the distribution of the $\beta_3$-AR, the
approach was extended to use a FRET-based reporter of cyclic GMP, showing that
a redistribution of the $\beta_3$-AR occurs in failing cardio-myocytes, too
\cite{Schobesberger2020}.

\subsection{Combining SICM and confocal FM}

\begin{figure}
  \includegraphics[width=\textwidth]{gr/correlating/sicm+cfm}
  \caption{%
    \textbf{Combining SICM and confocal fluorescence microscopy.}
    \textbf{a:} Schematic illustration of the correlating data acquired by combining SICM and confocal FM.
    \textbf{b:} 
    	\textit{Left:} Topography of a neuronal network.
    	\textit{Middle:} Fluorescence of the same neuronal network as depicted on the left.
    	\textit{Right:} Overlay of topography and fluorescence of two regions of interest (ROI 1, ROI 2) from
    	the left and middle image.
    Reprinted with permission from ref \cite{Novak2013}.
  }
  \label{fig:sicm+cfm}
\end{figure}

Similar to a SICM and wide-field combination, a combination of SICM and
confocal FM enables imaging of the same region and sample with both techniques
(Fig.~\ref{fig:sicm+cfm}a). The technical implementation involves mounting a
SICM onto an inverted confocal FM. The software used to control the two
microscopes does not have to be the same as both microscopes record data
independently from one another. However, as the recordings are not
simultaneous, pixel-by-pixel correlation is hardly, if at all, possible.

SICM and confocal FM have been combined for the first time by Korchev et al.
\cite{Korchev2000} in 2000. The group used this combination to verify the
volume measurements obtained by SICM. Just a year later, further development
of this combination enabled simultaneous acquisition of Ca\textsuperscript{2+}
imaging and topography data \cite{Shevchuk2001}.
% Afterwards, many different
% studies using a combination of SICM and confocal FM were published:

Seifert et al. \cite{Seifert2017} investigated the morphological activity of
platelets by time-lapse SICM and subsequently obtained confocal recordings of
cytoskeletal proteins of the investigated cells, allowing them to correlate
the activity of the platelets and the structure of the cytoskeleton. 

SICM has been shown to allow patch-clamp recordings from tiny neuronal
structures like synaptic boutons \cite{Novak2013}. The combination of SICM and
confocal FM allowed to first record an overview image and to identify active
synaptic boutons (Fig.~\ref{fig:sicm+cfm}b), which, in a second step, could be
patch-clamped.

A second study on neurons investigated, besides others, cargo transport in
neuronal dendrites \cite{Takahashi2019}. Here, the combination of
confocal FM and SICM provided a hint that the transport processes observed by
SICM were mediated by mitochondria.

In a study investigating the formation of human immunodeficiency virus-like
particles (VLPs) the correlated SICM-fluorescence confocal microscopy was used
to investigate topography changes together with die distribution of
fluorescently labeled viral proteins like Gag or Viral Protein R
\cite{Bednarska2020}. In contrast to investigating the virus formation with
TIRF microscopy \cite{Jouvenet2008}, the combination of SICM and confocal FM
allowed to investigate the apical side of the cells. It was found that the
release of VLPs is a lot faster than the release at the basal cell side as
known from TIRF investigations and furthermore, a potential impact on the
release time by the application of a fluorescent marker was found
\cite{Bednarska2020}.

% ich würde hier etwas zu potentiellen z-stacks + SICM sagen wollen, das hat
% ja meines Wissens nach noch niemand gemacht...
The applications reviewed above used only a single confocal layer. Yet,
a combination of SICM and $z$-stacking confocal FM could be used to yield more
correlating data. After imaging the entire volume that is scanned by SICM, the
data could be aligned and non-correlating data could be separated from
correlating data. If stage scanning would be omitted, the two measurements
could be performed simultaneously, but not synchronously.  


%% Wenn ich das Paper richtig verstehe, wurde hier aber kein konfokales FM
%% genutzt, oder? 
%
% Nashimoto et al. investigated whether the secretion of the von 
% Willebrand factor from living cells can be investigated by SICM without any
% additional labeling. , it is not always necessary to label the protein. Even with SICM alone the 
% detection of such a protein was possible, which was verified by comparing fluorescence and SICM 
% images \cite{Nashimoto2015}.

% \subsection{\emph{Indirect} combination of SICM and FM}

% Beside \emph{direct} combinations in one instrument and correlations, which are displaying 
% images obtained by SICM and fluorescence microscopy of the same cell/same region, also 
% \emph{indirect} correlations exist. They are showing images under comparable conditions and 
% extent the investigation to aspects, which could not be obtained by using one technique alone. 
% Such \emph{indirect} correlations were used for investigating the uptake of nanotubes 
% \cite{Lee2013}, nanoparticles \cite{Gesper2017} or for investigating healthy and failing 
% myocytes \cite{Lyon2009}. Nevertheless is it important to mention, that they could just giving 
% hints of correlations, it is not directly comparable with investigations of both in the same 
% cell. 

\subsection{Surface scanning}

Scanning surface confocal microscopy (SSCM) \cite{Gorelik2002a} is a
combination of SICM and scanning confocal microscopy (SCM) which enables
simultaneous SICM and FM imaging of the surface of the same sample. In SSCM,
the fluorescence data for each pixel is recorded at the position determined by
SICM for that pixel. Up to now, two slightly different setups have been proposed
(Fig.~\ref{fig:sscm_setup}) \cite{Gorelik2002a,Shevchuk2013}.

\begin{figure}
  \includegraphics[width=\textwidth]{gr/correlating/sscm_setup}

  \caption{%
    \textbf{Two setups for a SSC microscope.}
    \textbf{a:} Diagram of the combined setup with an $x,y,z$ sample scanner.
    \textbf{b:} Diagram of the combined setup with a $x,y$ sample scanner and
    a $z$-probe and objective scanner. 
    Reprinted with permission from ref \cite{Shevchuk2013}.
  }
  \label{fig:sscm_setup}
\end{figure}

In both setups, an SICM is mounted onto an inverted confocal microscope and
before simultaneous recording, the focus of the confocal FM objective is
aligned with the tip of the SICM probe via micrometer screws and the confocal
microscope's focusing mechanics. The
difference between the two setups is the way how the SICM pipette, the
confocal focus and the sample are positioned with respect to each other during
imaging. The
first setup (Fig.~\ref{fig:sscm_setup}a) keeps the 
SICM probe at a fixed position and scanning is performed by the sample
stage which positions the sample in all three dimensions.

In contrast, the second setup (Fig.~\ref{fig:sscm_setup}b) uses the sample
stage to position the sample in $x$- and $y$-direction only while objective
and probe are positioned by additional piezo elements in $z$-direction. Thus,
when the surface is detected by the SICM pipette,  the position of
the confocal spot has to be adjusted by the objective positioning piezo subsequently.

Both setups do not allow to use the controlling software of the confocal
microscope since the fluorescence data has to be recorded after the SICM
pipette has detected the surface. Thus, the SICM controlling soft- and
hardware had to be extended to at least allow reading from the detectors of
the confocal microscope. 

Switching from three-axis stage scanning to $z$-movement of the pipette
allowed for faster scan rates as the resonant frequency of the 3D
piezo-actor acted as a limiting factor. Additionally, usage of three
piezo-elements on the same platform led to cross-talk and consequentially
artefacts in the SICM data \cite{Shevchuk2013}.



Both setups keep the probe-objective distance constant and enable the
acquisition of a fluorescence image and a topography of the sample surface in
the same scan. Photobleaching is equalized for every point of the sample
allowing quantitative analysis of the FM data.

\begin{figure}
  \includegraphics[width=\textwidth]{gr/correlating/sicm+scm}
  \caption{%
    \textbf{Combining SICM and SCM.}
    \textbf{a:} Schematic illustration of the correlating data acquired by SSCM.
    \textbf{b:} Simultaneously acquired topographic and fluorescence data of fluorescent Cy3-labeled 
    VLPs adsorbed to the surface of Cos-7 cells. Reprinted with permission from ref \cite{Gorelik2002a}.
    \textbf{c:} SSCM recordings of fixed clathrin-GFP transfected Cos-7 cells. 
    	\textit{Left:} Cell surface topography. 
    	\textit{Center:} Fluorescence recording of the cell shown left. 
    	\textit{Right:} 3D overlay of topographical and fluorescent recordings. 
    Reprinted with permission from ref \cite{Shevchuk2008}.
    \textbf{d:} Live Cos-7 cells sequentially scanned by SSCM. Overlays of simultaneously obtained 
    topographic and fluorescence scans. Fluorescence imaging was started 2\,h 17\,min after the first 
    SICM scan as indicated by the red star. Fluorescent VLPs were added at 2\,h 40\,min. 
    Reprinted with permission from ref \cite{Gorelik2002}.
  }
  \label{fig:sicm+scm}
\end{figure}

As the SICM and SCM data sets are recorded simultaneously, all data points inherently have a
correlating data point in the other set. This immediate correlation, schematically illustrated in
Fig.~\ref{fig:sicm+scm}a, allows not only for correlating studies in fixed
samples (Fig.~\ref{fig:sicm+scm}c), but also studies of dynamics in live
samples (Fig.~\ref{fig:sicm+scm}d).\todo{Improve paragraph}

% Bis hier

SSCM has been used for investigating the dynamics of the interaction of
Cy3-labeled polyoma virus-like particles (VLPs) with Cos-7 cells
(Fig.~\ref{fig:sicm+scm}b,d) \cite{Gorelik2002a}, for the investigation of the
uptake of fluorescent nanoparticles in immortalized human alveolar epithelial
type 1 (AT1) cells and AT2 cells \cite{Kemp2008,Novak2014} and for investigating the
differentiation of stem cells \cite{Gorelik2008}. Furthermore, it has been
utilized in studies investigating endocytotic pathways \cite{Shevchuk2008}
(Fig.~\ref{fig:sicm+scm}c) and in a study revealing an alternative closing
mechanism for clathrin-coated pits by recording topography simultaneously with
the fluorescence of tagged actin-binding protein and the tagged
clathrin light-chain \cite{Shevchuk2012}. A further study on clathrin-mediated endocytosis revealed the effect of mutations in the Dynamin~2
gene on the maturation and internalisation of clathrin coated pits \cite{Ali2019}.

\subsection{Nanoinjection of fluorescently labeled molecules}

Nevertheless SICM does not necessary being used for investigating the topography of a cell, 
the nanopipette of a SICMic could be also used for the targeted deposition of molecules, like 
fluorescently-labeled single stranded-DNA molecules \cite{Ying2002,Hennig2015}. The nanopipette 
could be also used for nanoinjection of fluorescently labeled molecules into living cells 
\cite{Hennig2015a}. The SICM was therefore combined with wide-field and direct stochastic 
optical reconstruction microscopy (dSTORM), and mitochondria, actin and DNA could be successfully 
stained. One problem with the nanoinjection was that artifacts, due to the holes that were 
introduced into the membranes, could occur \cite{Hennig2015a}.

The delivery of fluorescently labeled molecules with the nanopipette could be also used for 
combining it with single-molecule diffusion investigations \cite{Bruckbauer2007, Bruckbauer2010}. 
In both studies the scanning ion conductance microscope was combined with TIRF investigations and 
the diffusion of single molecules over different regions at a spermatozoa were investigated. 


The SICM-nanopipette could be also used for investigating membrane rafts 
\cite{Simons1997,Pike2006} by coordinated manipulation of focal adhesions. The pipette was 
functionalized with fibronectin, gangliosides were fluorescently labeled, the pipette deformed the 
focal adhesion side and the behavior of the fluorescently labeled gangliosides were detected. This 
was seen as a hint, how membrane rafts reacted \cite{Fuentes2012}. 

Another example for a SICM-nanopipette based manipulation could be found in the study on the 
generation and distribution of plasma-derived extracellular vesicles (PEVs) on MDA-MB-231 and U87 
cells. Here the nanopipette were used for both, investigating the topography and for inducing the 
generation of the PEVs. Additionally the SICM was build up onto a confocal microscope, which could 
be used for the detection of Calcium \cite{Wang2020}.

% Dieser Abschnitt passt nicht so wirklich zu Fluoreszenz-Mikroskopie! Vielleicht noch so was 
% schreiben, wie „SICM kann nicht nur mit Fluoreszenz mikroskopischen Methoden kombiniert werden, 
% sondern kann auch mit … kombiniert werden!? Sonst kann der einfach auch weg. Mal gucken.

% By using a special illumination system consisting of a light emitting diode (LED) light ring, SICM 
% could also being combined with condenser-free phase contrast, darkfield and brightfield images and 
% even Rheinberg illuminations were possible \cite{Webb2014}.


%%% Local Variables:
%%% mode: latex
%%% TeX-master: "../manuscript"
%%% End:

\section{Correlating SICM and Super-Resolved Fluorescence Microscopy}
\label{sec:correlating-sicm-and-srfm}

The development of correlated
SICM and SRFM is still in its very early stages. The first combination of SRFM
and SICM has been the use of STORM to determine the number of fluorescent
molecules deposited via an SICM tip \cite{Hennig2015}. However, recently a
first proof-of-principle study combining STED and SICM has been published
\cite{Hagemann2018}.

In the latter, the topography of a group of fixed HeLa has been investigated
by SICM (Fig.~\ref{fig:SICM_STED1}a) and, subsequently, the distribution of
the cytoskeletal protein actin was determined by confocal and STED microscopy
(Fig.~\ref{fig:SICM_STED1}b) on a second instrument. The area of the sample
imaged by both methods is marked in Fig.~\ref{fig:SICM_STED1}c (SICM: yellow
rectangle, STED: gray rectangle).



\begin{figure}	
  \centering
    \includegraphics[width=\textwidth]{gr/correlating/Hagemann2018-F4}
      \caption{
      \textbf{Correlative SICM and STED recording of a group of HeLa cells.} 
      \textbf{a:} Height (aa) and slope representatin (ab) of a SICM recording
      of HeLa cells. \textbf{b:} Confocal (ba) and STED (bb) recording of the
      region marked  
      with a black dotted line in a. \textbf{c:}  Relative position of 
      the SICM (yellow rectangle) and the confocal/STED (gray rectangle)
      images with respect to  each 
      other. \textbf{d:} Magnification of the STED (da), confocal  (db), SICM
      (dd) and slope (de) images. The magnified region is marked with a white 
      arrowhead in a and b. \textbf{df} and \textbf{dg} show an overlay of 
      the STED and the SICM data. 
      \textbf{Dc} shows the height profile along the white dotted line in dd.  Reprinted 
      with permission from Philipp Hagemann, Astrid Gesper, Patrick Happel: \emph{Correlative 
      stimulated emission depletion and scanning ion conductance microscopy.} ACS Nano 2018, 
      12, 5807-5815. Copyright 2018 American Chemical Society.}
  \label{fig:SICM_STED1}
\end{figure}

Most interestingly, a protrusion (white arrowhead in
Fig.~\ref{fig:SICM_STED1}a, b) extending from one cell was captured by both
methods (magnified in Fig.~\ref{fig:SICM_STED1}d). The confocal recording
showed a blurred spot within the protrusion (Fig.~\ref{fig:SICM_STED1}db),
while the STED recording revealed that actin was arranged as a ring-shaped
structure around two central spots (Fig.~\ref{fig:SICM_STED1}da). The authors
speculate that this structure might be involved in the attachement or
detachement of the cell from the cell culture dish. However, since the
recordings were performed on fixed cells, time-lapse data that supports this
speculation could not be obtained.

% Bis hier

% As mentioned in section~\ref{sec:correlating-sicm-and-fm}, SICM was also successfully combined 
% with dSTORM (see section~\ref{sec:smlm}) \cite{Hennig2015a}. The difference to the 
% SICM/STED microscopy correlation study was, that the SICM was in that study used for nanoinjecting 
% fluorescently labeled molecules -- ATTO 655-phalloidin -- into living U2OS cells. 

% Hier jetzt noch was zu SICM/SNOM!

%%% Local Variables:
%%% mode: latex
%%% TeX-master: "../manuscript"
%%% End:

\section{Potential Pitfalls}
\label{sec:pitfalls}

Combining SICM and FM or SRFM techniques leads to a variable quantity of 
correlating data depending on the overlap of data acquired by each technique.
Figure \ref{fig:CombinedMicroscopy} schematically illustrates implementations 
of combined microscopes and the respective resulting correlating data. 

\begin{figure}
  \sidecaption
  \includegraphics{gr/pitfalls/CombinedMicroscopy}%
  \caption{\textbf{Potential implementations of combined SICM and FM.} 
  			\textit{Left:} Schematic depiction of potential combined SICM 
  			and TIRF or standard wide-field microscopy (\textit{top}),
  			SICM and z-stacking (\textit{middle}) and SICM and surface
  			scanning (\textit{bottom}). 
  			\textit{Right:} Resulting correlating data of these combinations.}
  \label{fig:CombinedMicroscopy}
\end{figure}

A combination of SICM and wide-field microscopy techniques such as TIRF yields
two data sets, just as any other combination would. However, before these data
sets can be combined and examined for correlating data, it has to be ensured that 
a data point from one set is assigned a data point from the other set. As SICM 
can have significantly higher resolutions than wide-field microscopy, it is 
likely that one data point from the wide-field microscopy data set is assigned 
multiple data points from the SICM data set. To avoid this, the SICM pipette 
could be adjusted to yield a lower resolution comparable to the wide-field 
microscopy resolution. In case of a resolution mismatch the data sets have to be
aligned after imaging, as wide-field microscopy is not a scanning technique and
records all data at once instead of one another making it impossible to align the
microscopes and thus the data recording prior to imaging. Due to the resolution
mismatch this can prove difficult.

Once the data is aligned, it can be checked for correlations. As SICM traces the 
surface of the sample, it changes its position constantly in all 3 dimensions unless
the sample has a very flat surface and a change in z direction is not necessary.
In contrast, wide-field microscopy only records two dimensional data (in x and y
directions). This means, that data points from the two different sets only truly 
correlate, if the focus of the wide-field microscope is set to a focal plane (z)
in the same or nearly the same z position as the SICM. To achieve this, the SICM
should either not change its z position much, which is only reasonable for
relatively flat samples, or SICM data from different z planes has to be discarded.
Flat cell processes or extensions are an example for a structure, which could be 
investigated by a such a combination.

SICM and z-stacking FM techniques, when combined, yield more correlating data than
SICM and wide-field techniques as z-stacking microscopy as this FM technique
records data points in multiple z planes. After imaging these can be aligned with 
the SICM data set and non-correlating data can be separated from correlating data.
However, there is still a potential mismatch in resolution and the data has the 
processed after imaging to yield correlation. Yet, this implementation is suitable
for all, not only flat, samples.

Implementations of SICM and surface scanning microscopy automatically solve one
of the problems occuring when combining SICM with other FM techniques: Both 
microscopes are scanning techniques able to change their z position from one data
point to the next enabling an alignment before data recording. Therefore, all 
recorded data, if aligned correctly, can be assumed to be correlating.

Instead of having to worsen SICM resolution to match FM resolution, it might be 
reasonable to combine SICM and SRFM in one setup. Though, this implementation
comes with its own potential pitfalls, especially as due to the super resolution
of both techniques small deviations or disturbances are sufficient to ruin the 
alignment of the setup. Some of these potential pitfalls are discussed in the 
following:

To avoid some problems from the beginning, both the SRFM and SICM setups
should provide spatial resolutions in the same range. Also, if possible, one
software should be used to control both setups to prevent delays in the
imaging process due to transmitting information from one software to the other.

Some potential pitfalls can arise when aligning the SRFM and SICM
setups. There are two alignment options: The first one is aligning the pipette
tip and the laser beam only in the xy plane, e.g. for imaging fluorescent
proteins in the cytoskeleton and the cell membrane simultaneously. This is 
comparable, in terms of data recording, to the above mentioned combinations of
SICM and wide-field or z-stacking microscopy. 
The second one, surface scanning, is additionally aligning the tip and beam
in the z plane, e.g. to image cell membrane and fluorescent particles or proteins 
within the membrane simultaneously, as has been done in 
SSCM \cite{Gorelik2002a}\cite{Shevchuk2008}.

Especially the second option tracing the cell surface would require constant
focal plane adjustment of the optical setup. This might eventually lead to a
mismatch in the excitation and depletion beam superimposition.

It has to be put into consideration whether fine adjustment screws are enough
to align the pipette and the beam. It is apparently sufficient for aligning
SICM and confocal microscopes \cite{Gorelik2002a}\cite{Shevchuk2008}. 
In case it isn't accurate enough piezo elements might have to be used. 
However, the usage of 3 piezo elements mounted to the same platform, may it be
the pipette holder or sample stage, is reported to have caused cross-talk during
sample positioning leading to image artifacts in the range of 100 nm 
\cite{Shevchuk2013}. Hence, it would be better to uncouple xy piezos from 
the z piezo. Furthermore, z piezos come with an increased risk of piezo drift. 
Therefore, especially the sample stage should ideally avoid z piezo usage.

Both alignment options may lead to the pipette tip reflecting the beam and
thereby distorting the resulting recordings. In this case the pipette tip
would have to be retracted until it can't reflect the beam light anymore,
before a recording with the optical setup can be executed. Consequently, this
means the SRFM and the SICM recording can't be executed simultaneously but
successively.

While this might seem like a drawback extending the overall recording time, it
also solves another problem that comes with combining SRFM and SICM:
photo-bleaching and photo-damage of the sample. SRFM techniques like STED
microscopy come with a general risk of photo-bleaching and -damage as high
intensity lasers are used. Combining SRFM with SICM increases this risk,
as SICM is a much slower imaging technique. The pipette will not only have to
be moved in the xy plane but also in the z plane to avoid sample
contact. Hence, the imaging speed will mostly be dependent on the SICM
capillary speed especially in z direction. If the sample is illuminated by the
laser continuously during capillary approach and retraction, photo-bleaching
and -damage is likely to occur. Therefore, illuminating the sample right after
SICM surface detection, recording and pipette retraction for a short period of
time would not only result in reflection avoidance but also in less
photo-bleaching and -damage. In case, light reflection by the scanning
capillary doesn't occur, the sample could be illuminated right after surface
detection leading to a faster imaging speed of the combined setup.

The switching of the laser beams could either be done via mechanical
shutters. However, as mechanical shutters have a limited switching frequency,
it might be preferential to switch beams by switching akkusto-optical modulators.

Another problem stemming from capillary movement is the resonance of the
pipette resulting from it. Some groups have ensured resonance reduction by
using a v-groove mounting plate for the capillary instead of the conventional
patch clamp pipette holders \cite{Shevchuk2013}.

 	

%%% Local Variables:
%%% mode: latex
%%% TeX-master: "../manuscript"
%%% End:


\bibliographystyle{spphys}
\bibliography{literature}


\end{document}

% Here are some introductions how to use the class

Use the standard \verb|equation| environment to typeset your equations, e.g.
%
\begin{equation}
a \times b = c\;,
\end{equation}
%
however, for multiline equations we recommend to use the \verb|eqnarray| environment\footnote{In physics texts please activate the class option \texttt{vecphys} to depict your vectors in \textbf{\itshape boldface-italic} type - as is customary for a wide range of physical subjects}.
\begin{eqnarray}
\left|\nabla U_{\alpha}^{\mu}(y)\right| &\le&\frac1{d-\alpha}\int
\left|\nabla\frac1{|\xi-y|^{d-\alpha}}\right|\,d\mu(\xi) =
\int \frac1{|\xi-y|^{d-\alpha+1}} \,d\mu(\xi)  \\
&=&(d-\alpha+1) \int\limits_{d(y)}^\infty
\frac{\mu(B(y,r))}{r^{d-\alpha+2}}\,dr \le (d-\alpha+1)
\int\limits_{d(y)}^\infty \frac{r^{d-\alpha}}{r^{d-\alpha+2}}\,dr
\label{eq:01}
\end{eqnarray}

\subsection{Subsection Heading}
\label{subsec:2}
Instead of simply listing headings of different levels we recommend to let every heading be followed by at least a short passage of text.  Further on please use the \LaTeX\ automatism for all your cross-references\index{cross-references} and citations\index{citations} as has already been described in Sect.~\ref{sec:2}.

\begin{quotation}
Please do not use quotation marks when quoting texts! Simply use the \verb|quotation| environment -- it will automatically be rendered in line with the preferred layout.
\end{quotation}


\subsubsection{Subsubsection Heading}
Instead of simply listing headings of different levels we recommend to let every heading be followed by at least a short passage of text.  Further on please use the \LaTeX\ automatism for all your cross-references and citations as has already been described in Sect.~\ref{subsec:2}, see also Fig.~\ref{fig:1}\footnote{If you copy text passages, figures, or tables from other works, you must obtain \textit{permission} from the copyright holder (usually the original publisher). Please enclose the signed permission with the manuscript. The sources\index{permission to print} must be acknowledged either in the captions, as footnotes or in a separate section of the book.}

Please note that the first line of text that follows a heading is not indented, whereas the first lines of all subsequent paragraphs are.

% For figures use
%
%\begin{figure}[b]
%\sidecaption
% Use the relevant command for your figure-insertion program
% to insert the figure file.
% For example, with the graphicx style use
%\includegraphics[scale=.65]{figure}
%
% If no graphics program available, insert a blank space i.e. use
%\picplace{5cm}{2cm} % Give the correct figure height and width in cm
%
%\caption{If the width of the figure is less than 7.8 cm use the \texttt{sidecapion} command to flush the caption on the left side of the page. If the %figure is positioned at the top of the page, align the sidecaption with the top of the figure -- to achieve this you simply need to use the optional %argument \texttt{[t]} with the \texttt{sidecaption} command}
%\label{fig:1}       % Give a unique label
%\end{figure}


\paragraph{Paragraph Heading} %
Instead of simply listing headings of different levels we recommend to let every heading be followed by at least a short passage of text.  Further on please use the \LaTeX\ automatism for all your cross-references and citations as has already been described in Sect.~\ref{sec:2}.

Please note that the first line of text that follows a heading is not indented, whereas the first lines of all subsequent paragraphs are.

For typesetting numbered lists we recommend to use the \verb|enumerate| environment -- it will automatically rendered in line with the preferred layout.

\begin{enumerate}
\item{Livelihood and survival mobility are oftentimes coutcomes of uneven socioeconomic development.}
\begin{enumerate}
\item{Livelihood and survival mobility are oftentimes coutcomes of uneven socioeconomic development.}
\item{Livelihood and survival mobility are oftentimes coutcomes of uneven socioeconomic development.}
\end{enumerate}
\item{Livelihood and survival mobility are oftentimes coutcomes of uneven socioeconomic development.}
\end{enumerate}


\subparagraph{Subparagraph Heading} In order to avoid simply listing headings of different levels we recommend to let every heading be followed by at least a short passage of text. Use the \LaTeX\ automatism for all your cross-references and citations as has already been described in Sect.~\ref{sec:2}, see also Fig.~\ref{fig:2}.

For unnumbered list we recommend to use the \verb|itemize| environment -- it will automatically be rendered in line with the preferred layout.

\begin{itemize}
\item{Livelihood and survival mobility are oftentimes coutcomes of uneven socioeconomic development, cf. Table~\ref{tab:1}.}
\begin{itemize}
\item{Livelihood and survival mobility are oftentimes coutcomes of uneven socioeconomic development.}
\item{Livelihood and survival mobility are oftentimes coutcomes of uneven socioeconomic development.}
\end{itemize}
\item{Livelihood and survival mobility are oftentimes coutcomes of uneven socioeconomic development.}
\end{itemize}

% \begin{figure}[t]
% \sidecaption[t]
% % Use the relevant command for your figure-insertion program
% % to insert the figure file.
% % For example, with the option graphics use
% \includegraphics[scale=.65]{figure}
% %
% % If no graphics program available, insert a blank space i.e. use
% %\picplace{5cm}{2cm} % Give the correct figure height and width in cm
% %
% %\caption{Please write your figure caption here}
% \caption{If the width of the figure is less than 7.8 cm use the \texttt{sidecapion} command to flush the caption on the left side of the page. If the figure is positioned at the top of the page, align the sidecaption with the top of the figure -- to achieve this you simply need to use the optional argument \texttt{[t]} with the \texttt{sidecaption} command}
% \label{fig:2}       % Give a unique label
% \end{figure}

\runinhead{Run-in Heading Boldface Version} Use the \LaTeX\ automatism for all your cross-references and citations as has already been described in Sect.~\ref{sec:2}.

\subruninhead{Run-in Heading Boldface and Italic Version} Use the \LaTeX\ automatism for all your cross-refer\-ences and citations as has already been described in Sect.~\ref{sec:2}\index{paragraph}.

\subsubruninhead{Run-in Heading Displayed Version} Use the \LaTeX\ automatism for all your cross-refer\-ences and citations as has already been described in Sect.~\ref{sec:2}\index{paragraph}.
% Use the \index{} command to code your index words
%
% For tables use
%
\begin{table}[!t]
\caption{Please write your table caption here}
\label{tab:1}       % Give a unique label
%
% Follow this input for your own table layout
%
\begin{tabular}{p{2cm}p{2.4cm}p{2cm}p{4.9cm}}
\hline\noalign{\smallskip}
Classes & Subclass & Length & Action Mechanism  \\
\noalign{\smallskip}\svhline\noalign{\smallskip}
Translation & mRNA$^a$  & 22 (19--25) & Translation repression, mRNA cleavage\\
Translation & mRNA cleavage & 21 & mRNA cleavage\\
Translation & mRNA  & 21--22 & mRNA cleavage\\
Translation & mRNA  & 24--26 & Histone and DNA Modification\\
\noalign{\smallskip}\hline\noalign{\smallskip}
\end{tabular}
$^a$ Table foot note (with superscript)
\end{table}
%
\section{Section Heading}
\label{sec:3}
% Always give a unique label
% and use \ref{<label>} for cross-references
% and \cite{<label>} for bibliographic references
% use \sectionmark{}
% to alter or adjust the section heading in the running head
Instead of simply listing headings of different levels we recommend to let every heading be followed by at least a short passage of text.  Further on please use the \LaTeX\ automatism for all your cross-references and citations as has already been described in Sect.~\ref{sec:2}.

Please note that the first line of text that follows a heading is not indented, whereas the first lines of all subsequent paragraphs are.

If you want to list definitions or the like we recommend to use the enhanced \verb|description| environment -- it will automatically rendered in line with the preferred layout.

\begin{description}[Type 1]
\item[Type 1]{That addresses central themes pertainng to migration, health, and disease. In Sect.~\ref{sec:1}, Wilson discusses the role of human migration in infectious disease distributions and patterns.}
\item[Type 2]{That addresses central themes pertainng to migration, health, and disease. In Sect.~\ref{subsec:2}, Wilson discusses the role of human migration in infectious disease distributions and patterns.}
\end{description}

\subsection{Subsection Heading} %
In order to avoid simply listing headings of different levels we recommend to let every heading be followed by at least a short passage of text. Use the \LaTeX\ automatism for all your cross-references and citations citations as has already been described in Sect.~\ref{sec:2}.

Please note that the first line of text that follows a heading is not indented, whereas the first lines of all subsequent paragraphs are.

\begin{svgraybox}
If you want to emphasize complete paragraphs of texts we recommend to use the newly defined class option \verb|graybox| and the newly defined environment \verb|svgraybox|. This will produce a 15 percent screened box 'behind' your text.

If you want to emphasize complete paragraphs of texts we recommend to use the newly defined class option and environment \verb|svgraybox|. This will produce a 15 percent screened box 'behind' your text.
\end{svgraybox}


\subsubsection{Subsubsection Heading}
Instead of simply listing headings of different levels we recommend to let every heading be followed by at least a short passage of text.  Further on please use the \LaTeX\ automatism for all your cross-references and citations as has already been described in Sect.~\ref{sec:2}.

Please note that the first line of text that follows a heading is not indented, whereas the first lines of all subsequent paragraphs are.

\begin{theorem}
Theorem text goes here.
\end{theorem}
%
% or
%
\begin{definition}
Definition text goes here.
\end{definition}

\begin{proof}
%\smartqed
Proof text goes here.
%\qed
\end{proof}

\paragraph{Paragraph Heading} %
Instead of simply listing headings of different levels we recommend to let every heading be followed by at least a short passage of text.  Further on please use the \LaTeX\ automatism for all your cross-references and citations as has already been described in Sect.~\ref{sec:2}.

Note that the first line of text that follows a heading is not indented, whereas the first lines of all subsequent paragraphs are.
%
% For built-in environments use
%
\begin{theorem}
Theorem text goes here.
\end{theorem}
%
\begin{definition}
Definition text goes here.
\end{definition}
%
\begin{proof}
%\smartqed
Proof text goes here.
%\qed
\end{proof}
%
\begin{trailer}{Trailer Head}
If you want to emphasize complete paragraphs of texts in an \verb|Trailer Head| we recommend to
use  \begin{verbatim}\begin{trailer}{Trailer Head}
...
\end{trailer}\end{verbatim}
\end{trailer}
%
\begin{question}{Questions}
If you want to emphasize complete paragraphs of texts in an \verb|Questions| we recommend to
use  \begin{verbatim}\begin{question}{Questions}
...
\end{question}\end{verbatim}
\end{question}
\eject%
\begin{important}{Important}
If you want to emphasize complete paragraphs of texts in an \verb|Important| we recommend to
use  \begin{verbatim}\begin{important}{Important}
...
\end{important}\end{verbatim}
\end{important}
%
\begin{warning}{Attention}
If you want to emphasize complete paragraphs of texts in an \verb|Attention| we recommend to
use  \begin{verbatim}\begin{warning}{Attention}
...
\end{warning}\end{verbatim}
\end{warning}

\begin{programcode}{Program Code}
If you want to emphasize complete paragraphs of texts in an \verb|Program Code| we recommend to
use

\verb|\begin{programcode}{Program Code}|

\verb|\begin{verbatim}...\end{verbatim}|

\verb|\end{programcode}|

\end{programcode}
%
\begin{tips}{Tips}
If you want to emphasize complete paragraphs of texts in an \verb|Tips| we recommend to
use  \begin{verbatim}\begin{tips}{Tips}
...
\end{tips}\end{verbatim}
\end{tips}
\eject
%
\begin{overview}{Overview}
If you want to emphasize complete paragraphs of texts in an \verb|Overview| we recommend to
use  \begin{verbatim}\begin{overview}{Overview}
...
\end{overview}\end{verbatim}
\end{overview}
\begin{backgroundinformation}{Background Information}
If you want to emphasize complete paragraphs of texts in an \verb|Background|
\verb|Information| we recommend to
use

\verb|\begin{backgroundinformation}{Background Information}|

\verb|...|

\verb|\end{backgroundinformation}|
\end{backgroundinformation}
\begin{legaltext}{Legal Text}
If you want to emphasize complete paragraphs of texts in an \verb|Legal Text| we recommend to
use  \begin{verbatim}\begin{legaltext}{Legal Text}
...
\end{legaltext}\end{verbatim}
\end{legaltext}
%
\begin{acknowledgement}
If you want to include acknowledgments of assistance and the like at the end of an individual chapter please use the \verb|acknowledgement| environment -- it will automatically be rendered in line with the preferred layout.
\end{acknowledgement}
%
\section*{Appendix}
\addcontentsline{toc}{section}{Appendix}
%
%
When placed at the end of a chapter or contribution (as opposed to at the end of the book), the numbering of tables, figures, and equations in the appendix section continues on from that in the main text. Hence please \textit{do not} use the \verb|appendix| command when writing an appendix at the end of your chapter or contribution. If there is only one the appendix is designated ``Appendix'', or ``Appendix 1'', or ``Appendix 2'', etc. if there is more than one.

\begin{equation}
a \times b = c
\end{equation}

%\input{references}
\end{document}

%%% Local Variables:
%%% mode: latex
%%% TeX-master: t
%%% End:
